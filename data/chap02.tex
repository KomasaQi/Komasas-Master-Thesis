% !TeX root = ../车云协同液罐车防侧翻行驶规划与控制研究.tex

\chapter{车云协同防侧翻场景任务体系架构设计}
\label{chap:architecture}

%(TODO:①第二章加入利用应急响应时间说明必须保证车端应急与防侧翻规划能力 ②加入)

随智能汽车技术与通讯技术发展的车路云一体化大背景下,液罐车及其相关的危化品运输企业等将通过通信网络与智能路侧设施、云端服务器等信息物理实体紧密连接,作为IVCPS的一部分:众多场景任务体系之一而存在。液罐车进一步的安全稳定运行需求将不再是仅仅通过对车辆本身进行系统工程甚至单纯的算法开发可以满足的,液罐车防侧翻即危化品公路安全运输的任务也逐步从单车的局部技术研究转向了车路云一体化体系工程解决方案的探索。本章基于IVCPS的本质及演进特点,提出面向场景基于模型与仿真的体系工程建模方法(Scenario-oriented Model-and-Simulation-based System-of-Systems Engineering, So-MBSoSE),并基于So-MBSoSE方法从战略、运行、服务、资源等方面建立了车云协同危化品运输场景任务体系的架构,利用架构在环方针验证了实现体系能力建立应满足的各方面指标,并在此过程中涌现出了分层规控保障安全的车云协同防侧翻架构。

\section{IVCPS分场景MBSoSE建模理论}

  本节主要针对IVCPS的组成与特征进行分析,得出其体系类型及适用于其架构建模的方法论。进一步地,提出满足上述特点的分场景MBSoSE建模方法,并对其流程进行详细的说明。

  \subsection{IVCPS总体架构与特征分析}

    IVCPS的本质属性及其演化规律是对其及其内涵的场景任务体系进行架构建模设计时必要考虑的关键。“体系”或“复杂大系统”(System of Systems, SoS)是指有相互关联并能产生新性质的独立系统的集合,其本身也是一种复杂的系统\cite{fengyimin2024soseframework}。体系由多个成员系统构成,表现出很多系统本身所不具备的特点,这使得其研究方法与系统有较大区别。

    为区分体系与系统,最早由Maier总结了令学术界较为认可的5条体系特征\cite{maier1996sosprinciples},包括成员系统的独立运作性、成员系统的地理分布性、体系自身的演化发展、体系管理的独立性、所谓“独木不成林”的体系整体的涌现性。Boardman继而又总结出了自治、连通、多样、归属和涌现这5个体系的特征\cite{boardman2006meaningof}也广为认可,与Maier的理论有大量共通之处,更多是服务于本体论的建模应用。综合Maier和Boardman各自提出的5条体系特征,可以总结体系被公认最多的3条特性就是成员系统可独立运行、体系整体涌现出新特征、体系随时间不断演化发展,IVCPS作为不断演进的全国性交通体系符合上述特点。

    \begin{figure}
      \centering
      \includegraphics[width=1.0\linewidth]{fig/chap2/sys_consis.jpg}
      % \caption*{图片注释解释说明}
      \caption{IVCPS整体架构与组成}
      \label{fig:ivcps_sys_consis}
    \end{figure} 

    并发涌现及场景任务的存在是IVCPS区别于以往讨论的其他体系的重要特点。车路云一体化系统建设与应用指南中详细描述了智慧公交、智慧环卫等共8个典型云控应用场景及其商业闭环\cite{caicv2024guideline},这些应用场景又可以具象化为更具体的高速公路队列预测性巡航\cite{lishuyan2022pccheavytrucks}、云控高速互通立交控制\cite{shijia2023cooperativemerging}、城市公交车绿波通行\cite{chenqien2023cpccbus}、城市路口协同规划\cite{hans2024}等具体的场景任务,这些场景任务的实例同时在交通系统中大量发生,但是物理层的资源是各场景共用的,这就需要在架构设计时考虑到场景任务间可能的交互,并提前在架构层面进行协调解决。

    体系的类型同样影响架构设计的方法,而IVCPS在演化过程中满足不同类型体系的特征。学术界较为公认的分类方法根据体系使命的有无及中央管理的强弱将体系分为4类:虚拟型、协作型、认同型和指向型\cite{fengyimin2024soseframework}。虚拟型体系无总体使命亦无中央管理相对松散,成员系统只完成各自目标;认同型体系拥有总体使命不完全的中央管理,但也依赖成员间的自主配合来实现目标,通常由现有资源整合而成,如交通系统\cite{heqiang2018sosepractice}。指向型体系通常导向式设计,并通过集中管理实现特定使命,所有成员系统完全服从统一指挥,常见于军事和航天领域。IVCPS不同场景任务体系的类型呈现多样化,除虚拟型外其他三种体系类型均有不同程度体现。但总体而言,随着其演进过程中ICV渗透率持续增加,低渗透率下多数场景任务内由于并非所有车辆可控,故场景任务体系基本为认同型,但总体IVCPS的各个场景协同相对欠缺,呈现协作型体系特点;全网联下多数场景任务内全部要素可控时,基本趋向指向型,但并非所有场景都可以很好协调,故IVCPS总体呈现认同型体系特点。

    对于虚拟型和协作型体系,由于其无中央管理,更适合于基于智能体的建模方法(Agent-based Modeling, ABM)\cite{baldwin2015simulation},即自底向上通过建模体系成员系统为智能体并进行模拟来涌现出整体行为,并根据其反馈进行其中某些成员系统的设计。如根据从交通仿真涌现出的交通情况设计信号灯配时。对于全权中央管理的指向型体系,更适合基于事件的建模方法(Event-based Modeling, EBM)\cite{hause2020uafple},即自顶向下通过架构的方式从设计阶段开始指导每个成员系统的设计符合总体要求。如从作战使命开始到舰艇、舰载机开发最终形成航母战斗体系。IVCPS的演进按照其主要驱动因素:车辆智能化的发展,分为了4个阶段,分别是当前阶段、低渗透率阶段、中高渗透率阶段,以及全智能网联阶段。对于IVCPS来说,低渗透下单个场景符合认同型,更适合采用ABM与EBM结合的方式设计架构,总体则需要ABM的涌现式建模反映多场景并行的效果以优化单个场景任务的设计;全网联下单个场景适合EBM的指向式建模,总体则遵循ABM与EBM结合的方式进行建模,既保证多场景任务设置符合预期,也能从仿真中得到反馈与验证。


    \begin{figure}
      \centering
      \includegraphics[width=1.0\linewidth]{fig/chap2/ivcps_arch_matrix.jpg}
      % \caption*{图片注释解释说明}
      \caption{IVCPS架构框架矩阵}
      \label{fig:ivcps_arch_matrix}
    \end{figure} 

    具体实施方面,如在低渗透率阶段(\SIrange{10}{30}{\percent})的车云协同危化品运输场景中,ABM自底向上将智能网联液罐车、HDV、路侧设备建模为独立智能体并定义自主决策逻辑,通过仿真呈现交通流交互特征;EBM自顶向下以云端为中央控制节点,通过事件驱动约束智能体交互规则,既保留自主性又实现关键任务引导;全智能网联阶段(>\SI{90}{\percent})的城市交通场景中,EBM主导通过中央云统一算法实现全网交通流优化,ABM 辅助模拟极端场景响应以验证中央策略鲁棒性、补充复杂交互局限性。So-MBSoSE 方法通过灵活融合 EBM 与 ABM 建模的思想,可适配 IVCPS 不同渗透率阶段的特性:在当前阶段及低渗透率阶段,针对 HDV 占比高、中央控制能力弱的特点,方法以 ABM 为主构建智能体模型模拟异构交通参与者的自主决策,同时通过 EBM 约束关键任务的核心交互规则,实现 “自主行为涌现 + 关键流程可控”;在中高渗透率阶段,随着可控车辆增多,方法增强 EBM 的顶层协同设计比重,通过分场景服务接口标准化支持多场景并行交互,并依托标准模型库快速复用共性模块;在全智能网联阶段,则以 EBM 主导全局优化,ABM 退化为极端场景验证辅助,通过多分辨率仿真实现从单一场景到全域协同的平滑拓展。这种动态适配机制确保方法能随 ICV 渗透率提升逐步调整建模重心,始终满足体系演进中的架构设计需求。

  \subsection{分场景MBSoSE建模理论}

    本小节描述建模语言、工具与所用框架,详细阐述用于车云协同液罐车防侧翻体系架构设计的分场景基于建模与仿真的IVCPS架构设计流程。

    \subsubsection*{(1)建模语言与框架}

      当前针对复杂体系的架构设计通常以国际对象管理组织(Object Management Group, OMG)所推行的统一架构框架(Unified Architecture Framework, UAF)\cite{hause2020uafple,hause2021securityviews,hause2021serviceviews}及其前身之一美国国防部架构框架(DoDAF)为基础进行裁剪\cite{abhaya2021ubm}并注入领域知识开展,如指挥作战体系\cite{qianqunyou2024uafurbancombat,lvweimin2021dodaf,cuiruijing2025uafindicator,lidelin2024uafcooperation}、航空航天体系等\cite{liminghua2021aerospace,huangran2024lunarserviceuaf}、跨国航空运输信息体系\cite{lisong2022uafairtransportation}等,并通常融合使命工程的思想开展分析\cite{wanglinyao2024missionengineeringreview,shiyujing2024missiontaskeval}。由于UAF先后融合了DoDAF与英国国防部架构框架(MoDAF)、北约架构框架(NAF)与企业架构(Enterprise Architecture, EA)等国际公认的国防与企业管理等领域的经典架构框架后,增长至了囊括架构管理与总结、战略、运行、服务、资源、人员、项目、防护、实际资源共9个视角,动机、分类、结构、连接、流程、状态、时序、信息、参数、约束、路线图、追溯性共12个方面的庞大体量,基本囊括了各领域考虑的范畴,并得到了如Sparx Systems Enterprise Architect、No Magic MagicDraw、IBM Rational Software Architect等一系列主流架构建模软件的底层配置支持。故肖雅月等\cite{qixiaojing2026mbsose}结合IVCPS性质,参考ICV领域的ICV-7S架构框架对UAF剪裁得到的IVCPS架构框架被用于架构建模。

      IVCPS架构框架如图~\ref{fig:ivcps_arch_matrix}所示共包含7个视角,和与UAF相同的12个方面,其中5个视角重定义自UAF并结合ICV-7S框架的的视角:战略、服务、信息物理资源、防护视角、标准化,1个重定义来自UAF的视角:运行视角。相比于UAF框架,IVCPS框架还重定义了1个来自ICV-7S的对其运行不可或缺的视角:安全视角,用于考虑体系级功能安全、预期功能安全。最后,该框架省略了日趋自动化的智慧交通系统中并不过分关注的人员配置视角以及有项目视角,将架构管理与总结视角合并入战略视角中,并将实际资源部分融合至成员系统的系统级架构设计中。

      7个视角确定了IVCPS构建的核心观察方向,而继承自UAF的12个方面提供了描述每个方向的模板。战略视角分析体系的发展阶段、能力,以及能力随着发展阶段的演变情况;运行视角捕捉运行方案与运行活动,梳理运行活动之间的逻辑顺序;服务视角通过服务的概念黑盒化运行活动,捕捉服务接口及服务规范;信息物理资源视角(简称“资源视角”)捕捉支持运行活动及服务实现的赛博或物理实体资源及其配置关系;防护视角捕捉战略、运行、服务、资源视角的信息安全相关风险及其防护措施,将防护措施落实到战略约束、信息安全防护相关服务、信息安全防护相关信息物理资源上;安全视角捕捉战略、运行、服务、资源视角的安全隐患及失效模式,对其进行安全可靠性分析并将缓解措施落实到战略约束、安全可靠性保障服务、安全可靠性保障资源上;标准化视角捕捉战略、运行、服务、资源视角中的元素涉及到的相关标准,并将其落实到战略约束、服务分类、资源分类视角中。 

      采用基于SysML1.5的UAF1.2框架元模型模拟IVCPS框架中相关元素,手动定义并添加安全视角至裁剪好的UAF框架内用于架构模型建立。

      \begin{figure}[htbp]
        \centering 
        \makebox[\linewidth]{\includegraphics[width=1.0\linewidth]{fig/chap2/time_series.jpg}}
        \caption{IVCPS-So-MBSoSE方法流程分视角时序拆解示意}
        \label{fig:ivcps_time_series}
      \end{figure}

      \begin{figure}
        \centering 
        \makebox[\linewidth]{\includegraphics[width=1.0\linewidth]{fig/chap2/specific_flowchart.jpg}}
        % \caption*{图片注释解释说明}
        \caption{IVCPS-So-MBSoSE方法流程分视图顺序详细拆解示意}
        \label{fig:ivcps_specific_flowchart}
      \end{figure} 

    \subsubsection*{(2)总体流程思路概述}

      如图~\ref{fig:ivcps_time_series}所示,IVCPS整体架构和单个场景包含相同类型的视图,其内容来自N个场景任务体系的架构相对应的视图,是场景任务体系架构的综合,在建模每个场景任务架构的对应视图的同时补充完善。在每个场景任务体系架构设计时分为4个阶段顺序进行:战略分析阶段、运行分析阶段、服务分析阶段以及资源分析阶段。
      战略分析阶段主要涉及战略层和信息物理资源层,目的是既从使命出发进行顶层设计,又考虑实际现有资源及其演进趋势,从中间向两边进行建模。运行分析阶段主要区分运行执行者并设计可能的总体运行方案,确保其逻辑可行并进行对比分析。服务分析阶段主要将通过服务的形式黑盒化相应运行活动,起到解耦运行层与资源层类似中间件的作用\cite{hause2021serviceviews},使架构具有更好的泛化性以应对敏捷开发过程中的变化及体系在不同阶段的演化需求。资源分析阶段进一步丰富资源间的结构、行为、信息、参数、约束等,以指导具体架构的落地实现。四个阶段中都有涉及到安全、防护与标准分析,其中战略层的相关分析结果输入战略约束视图;运行层的相关结果输入服务分类视图用于第三阶段创建满足安防要求及标准的相关服务;服务层相关分析结果用于创建安防及符合标准的相关资源;资源层同样将其本层的分析结果用于进一步迭代生成安防相关资源。


      每个阶段下自顶向下逐层分解的设计体现了EBM方法,而仿真验证也是IVCPS架构设计中非常重要的一环,其验证了逻辑与参数指标的合理与可行性,同时可以涌现出上层架构设计有时无法考虑到的成员系统间的相互作用以提供架构优化指导,体现了ABM思想。在运行分析阶段通过对运行方案的逻辑方案可以确认其运行逻辑按预想进行;服务分析阶段包含逻辑仿真与验证蒙特卡洛(Monte Carlo, MC)仿真。MC仿真通常用于在具有参数已知概率分布的情况验证体系级MoS (Measurement of Success)或成员系统级的MoEs (Measurement of Effects)\cite{qinchangmao2021remoteexpert}。在本阶段分析中用于描述服务规范的指标是否能满足战略能力的测度要求。在资源层同样包括逻辑仿真与MC仿真,此外还有场景任务体系参数仿真。其中逻辑仿真用于检验资源行为的逻辑通顺性;MC仿真用于进一步判断满足战略能力测度的服务规范是否能被其相关支撑资源所满足;场景任务体系参数仿真通过体系层简化的成员系统间的仿真(如车辆采用运动学模型,考虑简单的时延条件等)来在体系层面初步验证架构设计的合理性。由于多场景任务并行的特征,在同一个交通区域内可能有数个不同场景任务同时进行并存在相互干涉,同一个路口多个方向的公交车有绿波通行需求且路口存在车云协同规划与控制。故需要在多场景体系架构搭建完成后进行联合参数仿真,以对场景间配合进行验证并将结果用于各场景任务体系架构的优化。
      场景任务体系当中某些关注的成员系统或其组件系统,如智能网联车辆的网联域,如有需要可以对其在体系级分析的基础上进一步展开进行系统架构设计。系统架构采用RFLP方法\cite{dassault2012strategicse}。架构设计过程可以分为3个阶段,分别是:需求分析与上下文映射阶段、功能与逻辑架构构建阶段、信息物理架构与参数仿真阶段。系统架构部分在本研究中不进行展开。
      本方法场景任务循环体系架构构建流程如图~\ref{fig:ivcps_specific_flowchart}所示,详细描述了针对某场景任务体系进行的一次架构循环迭代中如何按照不同的视图顺序构建架构,数字序号标明了每个不同阶段下各个视图的建模顺序,同时箭头说明了输入输出依赖关系。



\section{车云协同液罐车防侧翻行驶架构建模}

本章以图~\ref{fig:ivcps_specific_flowchart}所示的So-MBSoSE方法流程为指导,构建车云协同液罐车防侧翻/车云协同危化品运输(Cloud-based tank truck rollover prevention, CT2ROP)场景任务体系架构。

  \subsection{CT2ROP架构概览}
  车云协同方案借助IVCPS实现的广域感知与并行算力的优势将动静态交通信息(道路结构、周车状态与历史轨迹等)用于进行长时域预测性规划,以避免极端工况的发生,并进一步提升车辆自身应对可能的突发情况的能力,保证液罐车运行安全性的前提下提高效率。CT2ROP场景是IVCPS并行交互场景其中重要的组成部分,目前正向IVCPS演进阶段中的低渗透率阶段演进。

  \begin{figure}
    \centering  
    \makebox[\linewidth]{\includegraphics[width=1.0\linewidth]{fig/chap2/ct2rop_concept.png}}
    % \caption*{图片注释解释说明}
    \caption{车云协同危化品运输场景概念架构}
    \label{fig:ct2rop_concept}
  \end{figure} 

  本场景内仿真验证主要通过运行层、服务层、资源层三级架构实现,以充分体现ABM和EBM的协同。运行层以战略能力要求为输入,通过时序图开展逻辑仿真并生成不同运行测试样例,输出运行方案修正建议,直至功能逻辑及输入输出完整无错。服务层基于战略层导出的服务约束和接口定义,先通过时序图验证服务交互逻辑,再利用MC仿真量化分析服务参数的概率分布,输出服务规范修正方案以指导资源层优化资源配置。资源层根据服务层确定的资源需求及成员系统物理参数,在MATLAB/Simulink中构建车辆-道路-云端联合模型,开展场景级参数仿真。

  根据图~\ref{fig:ivcps_specific_flowchart}所示流程以CT2ROP场景为本次迭代目标场景所建立的场景及IVCPS整体架构流程概览如图~\ref{fig:arch_process}所示,包含了7个视角下的架构模型,并主要按照蓝色粗实线箭头的顺序开展。

  \begin{figure}
    \centering  
    \makebox[\linewidth]{\includegraphics[width=1.0\linewidth]{fig/chap2/arch_process.jpg}}
    % \caption*{图片注释解释说明}
    \caption{车云协同危化品运输场景下IVCPS架构设计流程}
    \label{fig:arch_process}
  \end{figure} 

  \subsection{CT2ROP架构详细设计}
  \label{sub:ct2rop_arch_detail}

    本节展示CT2ROP场景任务体系的部分详细架构设计,并给出基于其得出的对于后续车云协同方法设计有益指导性结论与指标。

    \subsubsection*{(1)场景战略分析}

      战略分析阶段首先如上节所述对场景进行定义,并在其基础上分析了CT2ROP场景的战略动机及体系演进阶段,如图~\ref{fig:motivation}和图~\ref{fig:stage}所示;其次分析了现有资源结构,如图~\ref{fig:resource_model}所示;最后依据现有资源及战略阶段与目标,分析了体系安全风险并依据其推导出体系应具备的能力及其演进,如图~\ref{fig:safe_capability}所示,将分析结果用于各能力的战略约束中。

      如图~\ref{fig:motivation}所示,战略动机分析从现有驱动因素提出体系目标,并分析促进其实现的机遇以及产生驱动因素的内在挑战。CT2ROP场景任务体系受“更加安全高效的危化品运输”及“智能汽车信息物理系统建设”两大核心使命任务驱动,需解决车液耦合、单车局限与人因失误三大挑战,共同促成了这一场景任务体系的形成。

      \begin{figure}
        \centering 
        \makebox[\linewidth]{\includegraphics[width=1.0\linewidth]{fig/chap2/motivation.png}}
        % \caption*{图片注释解释说明}
        \caption{CT2ROP战略动机}
        \label{fig:motivation}
      \end{figure} 

      如图~\ref{fig:stage}所示,CT2ROP场景作为IVCPS的重要组成部分之一,遵循与IVCPS相同的演化路径,从当前阶段开始,经历低渗透率阶段,逐步提升到中高渗透率阶段,最终实现全智能网联阶段。在不同阶段对相同能力的处理方法不同,本文主要聚焦于低渗透率阶段的安全相关内容。

      战略动机与体系演进阶段分析过后需立足于现有资源,并分析资源的演化以更好支撑后续的能力分析。如图~\ref{fig:resource_model}所示,各场景场景任务体系及IVCPS总体进行组织,主要分为智能车辆、云控基础平台与云控应用平台、路侧基础设施、相关支撑平台等,后续阶段资源是先前阶段资源的继承与发展。不同场景任务中云端与路侧设施资源是共享的,共同打通了云服务的数据采集基座,而CT2ROP场景任务涉及到的特殊车辆类型液罐车、特殊组织实体危化品运输平台等是非共用的。在第四阶段,将会进一步根据根据服务、运行映射补全运行安全与信息安全相关软硬件资源,并将其与安全控制(Security Control)措施进行追溯以保证对于安全需求的覆盖。

      \begin{figure}
        \centering 
        \makebox[\linewidth]{\includegraphics[width=1.0\linewidth]{fig/chap2/stage.png}}
        % \caption*{图片注释解释说明}
        \caption{CT2ROP战略演进阶段}
        \label{fig:stage}
      \end{figure} 

      \begin{figure}
        \centering 
        \makebox[\linewidth]{\includegraphics[width=1.0\linewidth]{fig/chap2/resource_model.png}}
        % \caption*{图片注释解释说明}
        \caption{CT2ROP资源分类与结构}
        \label{fig:resource_model}
      \end{figure}

      CT2ROP场景任务体系的构建,需以液罐车运输领域的事故现状与核心风险为战略环境支撑,结合长期事故统计数据明确体系建设的靶向性与紧迫性,并对安全风险类型与核心验证场景的分析提供支持。从区域分布来看,液罐车危化品运输事故呈现明显的集中性特征,2004-2021年多时段统计数据均显示,山东、江苏、浙江始终处于事故高发省份行列,不同时段叠加广东、河南、湖北、陕西、安徽等省份,其中2013-2017年多统计渠道显示该区域集群事故量最高可达2075起,2016-2021年6年间高发省份事故总量达2020起,区域运输安全压力突出,成为体系需重点覆盖的核心区域。从运输道路环境来看,高速公路与一级公路是液罐车事故的主要发生场景,2004-2011年该类道路事故占比达\SI{60}{\percent},2013-2017年占比维持在\SI{58}{\percent},其次为二级公路与城市道路;路段层面,普通路段事故发生量占比最高,2004-2011年达\SI{62}{\percent},2013-2017年升至\SI{75.9}{\percent},而出入口匝道、下坡转弯路段、城市交叉口及桥梁、涵洞、收费站等特殊路段,因路况复杂、操作要求高,成为事故发生率最高的危险路段,这对体系的道路环境感知与动态适配能力提出了明确要求。事故成因与充装介质进一步凸显了防侧翻体系的核心价值,从事故诱因来看,车辆碰撞、追尾及侧翻是主要类型,2004-2011年车辆碰撞占比达\SI{38.7}{\percent}、单方面翻车占比\SI{23.6}{\percent},2013-2017年追尾与侧翻占比升至\SI{44.7}{\percent},2016-2021年更是高达\SI{62.7}{\percent},侧翻风险持续攀升且成为核心安全隐患;充装介质方面,液化石油气、液化天然气、汽油、柴油等危化品液体货物事故占比达\SI{70}{\percent}以上,是体系需重点防控的危险介质。此外,液罐车运输场景中,汽柴油等介质需从石油精炼厂转运至各地加油站,部分车辆在多站点卸载后处于半载状态,进一步提升了侧翻风险,这一运输流程特征也为车云协同防侧翻规划与控制提供了明确的场景导向,倒逼体系通过云平台协同感知、动态规划与实时控制,破解区域集中、道路复杂、诱因突出、场景特殊带来的运输安全难题。

      结合上述分析,运输过程中的风险可以可以依据其在事故逻辑链条中的位置分类为四大类型:根源性风险、诱发性风险、状态性风险以及后果性风险。根源性风险是先天/基础层面的缺陷,是所有其他类型风险的底层诱因,与软硬件设计、人员配置与培训、管理缺陷等有关;诱发性风险在运输过程中包括外部扰动/操作失误/协同链路异常等,其中罐体内液体晃动为液罐车特有,直接触发状态性风险,与人员操作、道路交通环境、自然环境、车云协作等有关;状态性风险在运输过程中包含车辆/罐体/车云协同的异常状态,若不及时干预将进一步发展为后果性的事故,如车辆动力学失稳包括侧翻、摆振、侧滑、折叠,含预失稳状态如侧倾角超标等,车载设备失效如结构失效中的爆胎、断轴、悬架损坏等,罐体状态异常如液位/压力超标、罐口密封松动等,车云协同链路异常如通信中断、决策时延过大、车端数据上传失败、指令下发错误等;最终的后果性风险来源于未及时处理的状态性风险,直接体现在最终事故形式上,为体系需规避的最终结果,其包涉及自车、周车、环境与各类交通参与人员。

      \begin{figure}
        \centering 
        \makebox[\linewidth]{\includegraphics[width=1.0\linewidth]{fig/chap2/safe_capability.png}}
        % \caption*{图片注释解释说明}
        \caption{CT2ROP安全与能力分析}
        \label{fig:safe_capability}
      \end{figure} 

      针对四类风险的特点使得对其进行安全控制的核心逻辑不同,基本思路为:根源性风险-源头防控、诱发性风险-过程预警、状态性风险-实时干预、后果性风险-应急处置。根源性风险需通过技术、管理与人员三类安全控制从源头消除或降低缺陷,建立体系基础保障,如在车辆/罐体设计阶段开展侧翻动力学仿真与测试、采用被动安全措施以及算法部署前在云端进行详细验证。诱发性风险需在过程监测基础上提前预警并主动规避,减少状态性风险的触发,控制措施包括车端自主对车辆与罐内风险感知与预警、车云协同风险预警与预测性决策、车云通信冗余保障、运输路线规避恶劣环境规避与自动驾驶及稳定性辅助操作。状态性风险需实时识别与干预,并保证失效冗余,阻止其发展为事故,其中核心在于车辆实时稳定性干预与失效应急预案的保障。后果性风险中事故已经发生,需应急处置进行损失控制,防范次生事故如危化品泄露导致的环境污染、人员中毒与起火爆炸等,为此应发挥车云协同的力量并加强车辆与罐体的被动安全防护,最终需溯源事故并实现体系的迭代演进。本研究主要从车云协同角度出发,聚焦主动安全,实现算法仿真测试、车端自主预警、车云协同预警、规控降级与记录反馈、紧急制动与避险、实时稳定性控制六大主动安全控制。

      经过上述分析,为实现所关注的主动安全控制,需图~\ref{fig:safe_capability}中的能力分析部分,需建立包含态势感知与风险识别、风险预测与预防性干预、风险控制与稳定性保障、鲁棒性与安全运营四大核心能力,的危化品公路运输主动安全保障的战略能力,其每项核心能力包括具体一系列子能力以支持最终实现侧翻、近侧翻事故率下降、紧急换道/制动频率下降,在通信/感知退化下仍保持安全并在保障安全前提下尽可能不损失运输效率的期望效果。相关能力分别对应相关测度指标,其一方面来源于先验知识,另一方面来源于安全、防护与标准化的分析,指标反映在战略约束中。

      对战略场景的分析也是战略分析阶段的重要一环,如图~\ref{fig:safe_capability}中的实际环境(Actual Environment)部分所示,CT2ROP场景任务体系的战略环境即将向低渗透率下的危化品运输环境演进。在该环境下,道路基础设施智能化水平覆盖部分交通设施,其中主要高速公路全路段、城市区域重点路段和路口实现路侧感知设备全覆盖,考虑跨省级持续数日的危化品运输任务,最大交通流量可达到大于等于\SI{80}{\percent}的道路承载能力。

    \subsubsection*{(2)场景运行分析}

      运行分析阶段通过梳理运行方案、拆分运行执行者及其结构,并如构建运行过程的功能架构及逻辑活动实现战略能力。上述功能架构及逻辑活动通过逻辑仿真进行梳理,并在安防与标准分析后将运行活动重新追溯到战略能力以保证完整覆盖。

      根据四大能力,可以划分运营监管、道路基础设施、协同支撑、运输车辆四大运行执行者,结合现有资源视图,分别可以进一步泛化为智慧危化品运输与指挥、路侧智能设施、边缘云加地图平台和交管平台、智能网联液罐车几个更为具象化的运行执行者,如图~\ref{fig:operation_model}所示,相关运行关运行执行者共同组成了CT2ROP运行架构。几大执行者构成的高层级概念图梳理实现战略能力的一种运行方案,在最终确定的运行方案中,智能网联液罐车作为危化品运输的执行者,受危化品运输企业调度并接受云控平台的建议以主动预警规避风险和进一步的事故。云控平台通过路侧设备得知道路交通参与者的结构化感知信息,并通过交管平台和地图平台获取准静态和静态的交通信息,如施工信息与地图信息。

      \begin{figure}
        \centering 
        \makebox[\linewidth]{\includegraphics[width=1.1\linewidth]{fig/chap2/operation_model.png}}
        % \caption*{图片注释解释说明}
        \caption{CT2ROP运行分类及结构分析}
        \label{fig:operation_model}
      \end{figure} 


      \begin{figure}
        \centering 
        \makebox[\linewidth]{\includegraphics[width=1.0\linewidth]{fig/chap2/operation_sim.png}}
        % \caption*{图片注释解释说明}
        \caption{CT2ROP运行功能与逻辑行为及其仿真}
        \label{fig:operation_sim}
      \end{figure} 



      如图~\ref{fig:operation_model}所示的CT2ROP内部运行连接图是功能结构的一部分,展示了运行架构当中各个运行执行者之间的交互,在搭建完功能架构后自动生成。如图~\ref{fig:operation_sim}所示,运行功能架构中适用于各个场景的运行活动将在IVCPS总体功能架构当中被描述,如更新实时交通孪生,而各个场景特有的运行活动将组织在其内部子文件夹下的子视图中,如CT2ROP运输功能活动。同时,各个运行执行者的内部状态机被建立以反映其状态变化,并添加状态转换及处于某状态时的活动,每个活动由更进一步的活动图描述。在运行过程中,危化品运输指挥与控制、边缘云云控基础平台、智能网联液罐车是最直接的执行者,分别实现静态调度、低频引导与高频执行。在液罐车运行状态中,详细描述了单车与协同两种模式的切换,而危化品运输指挥控制状态中则描述了运输任务的创建与结束过程。

      逻辑仿真通过时序图作为多个运行执行者之间交互的脚本,其相互间传递的信号及信号包含的信息用于各个运行执行者状态的转换及相应运行活动的执行;多个运行执行者之间的信息传递被同步显示在内部块图中以可视化运行过程。通过运行层的逻辑仿真梳理了运行逻辑,并初步确定运行信息的传递路径以为服务层进一步的细化进行了准备。


    \subsubsection*{(3)场景服务分析}

      服务分析阶段首先根据运行活动进行了服务定义,其次进行了服务流程的梳理得到服务功能与逻辑架构,最后在安全防护与标准分析后完善相关资源并完善服务约束及其追溯。

      细化服务约束后的服务定义如图~\ref{fig:service_model}所示,IVCPS总体中包含路测感知信息提供服务、地图信息提供服务、车云通信服务等可以用于各个场景的共性服务,CT2ROP场景任务体系下则包含了危化品运输车辆任务调度服务、车端防侧翻规划与控制服务云端预测性轨迹规划与液罐车装卸载的特有服务;每个服务中的指标通过对战略能力相应测度指标的支撑以及安全防护与标准分析得到,体现在服务约束图中。服务将运行活动中可重用的部分抽象黑盒化,被服务解耦后的运行层与资源层可以在不改变服务接口的情况下独立进行调整与演进,如运行方案中的逻辑适应长时期的发展不局限于具体实现方式,而资源的更新迭代却很快,那么只要保持服务接口的输入不变,就可以用更新的资源支持同样的防侧翻所需的运行活动。服务与运行活动的矩阵追溯关系映射辅助维护上述映射,并确保进一步到四大核心战略能力及其子能力的间接追溯。

      \begin{figure}
        \centering 
        \makebox[\linewidth]{\includegraphics[width=1.0\linewidth]{fig/chap2/service_model.png}}
        \caption*{(a)服务约束视图下定义的CT2ROP专有服务及IVCPS通用服务 (b)所有服务与运行活动间的追溯关系}
        \caption{CT2ROP服务定义及其与运行活动追溯}
        \label{fig:service_model}
      \end{figure} 


      CT2ROP服务功能、逻辑架构的建立及逻辑仿真的过程图~\ref{fig:service_sim}所示,服务架构囊括了场景任务必要的服务,并通过服务流程图展示服务之间的交互关系。CT2ROP整体服务逻辑展示了服务架构实现功能时的主要活动逻辑关系,其中每个活动都更详细地展开为由服务架构内部所包含子服务相互交互共同完成的服务活动流程流图。在完成服务层的功能与逻辑架构后,基于其生成了服务架构内部的运行信息交互关系,得到定义清晰的服务接口。

      \begin{figure}
        \centering 
        \makebox[\linewidth]{\includegraphics[width=1.0\linewidth]{fig/chap2/service_sim.png}}
        % \caption*{图片注释解释说明}
        \caption{CT2ROP服务功能与逻辑及其仿真}
        \label{fig:service_sim}
      \end{figure} 

      服务层的参数级仿真用于验证服务规范是否满足战略能力的相关测度。如图~\ref{fig:service_time_sim}所示,“低渗透率下的危化品运输”战略能力中包含应急响应时间测度指标,描述了当道路突发危险情况时从感知到最终应对危险的决策与规划被执行器执行的时间,是战略能力众多安全相关测度之一。该时延取决于车端感知、路端感知、车端决策规划、云端决策规划、车云通信时延、路云通信时延等多项服务的时间信息,通常可以用先验的分布描述,如车云通信时延可以被描述为对数正态分布,而云端决策通常分为多阶段可用伽马分布描述,其他决策规划与有线通信等场景可近似简化为常数\cite{qixiaojing2026mbsose}。

      战略约束层关于应急响应时间测度指标的约束被用于作为MC仿真结果的检验标准。在MC仿真分析模块中建立参数图,描述了从各个相关服务中提取服务规范参数、参数输入MATLAB中进行实际测度计算、返回实际测度并与战略约束要求进行比对以判断是否满足要求的过程。仿真验证结果通过柱状图可视化并反映在仿真实例表中。如图~\ref{fig:service_time_sim}最后的仿真实例表展示了两次各迭代1000次的MC仿真,通过对服务规范指标进行调整使得指标的超调比例和均值都符合标准,经过优化后的服务规范使得平均应急响应时间小于战略测度中\SI{100}{ms}的标准。通过上述方式进行架构中所有服务规范的指标进行设置,详细结果见建模文档\cite{KomasaQi2025CT2ROP_repo}。

      \begin{figure}
        \centering 
        \makebox[\linewidth]{\includegraphics[width=1.0\linewidth]{fig/chap2/servie_time_sim.png}}
        % \caption*{图片注释解释说明}
        \caption{服务阶段应急响应时间蒙特卡洛仿真}
        \label{fig:service_time_sim}
      \end{figure} 


    \subsubsection*{(4)场景资源分析}

      资源分析阶段根据运行与服务阶段进一步明确的结构并结合安全与防护需求,进一步完善在第一阶段战略分析阶段中得出的资源结构,最后通过资源的功能与逻辑构建描述资源实现服务的全流程,并完成相关追溯工作。如图~\ref{fig:safety_model}所示,详细描述了第一阶段IVCPS总体资源的结构及CT2ROP场景资源对其的继承与丰富。为满足战略阶段的主动安全控制措施,分别设计资源缓解(Resource Mitigation),建立云端高效高保真车液耦合仿真平满足算法仿真测试需求,为此需构建高保真液罐车动力学仿真模型以及高保真仿真工况与仿真器;通过构建由智能液罐系统和车载通信终端系统组成车辆状态监控与传感系统满足车辆自主预警的需求;通过实时孪生与预测性决策系统实现车云协同预警,需要构建云端考虑侧翻特性的预测性运动规划、云端考虑周车不确定性交互的预测性行为决策程序以及云端周车概率模型辨识程序;通过智能网联液罐车整体满足自动驾驶操作;利用车端防侧翻规划与车辆日志记录系统和车辆稳定性控制系统分别实现规控降级与记录反馈以及实时稳定性控制,其中分别包含了车端高频防侧翻避障规划程序和车端轨迹跟踪抑晃防侧翻控制程序,全部在智能网联液罐车的计算平台系统上实现。而对于信息安全防护的四大类信息安全隐患,同样采用资源缓解的方式进行防护设计,如车云通信安全保障资源包含车云通信加密模块与车云通信认证模块,确保车云通信的机密性与完整性;边缘云安全保障资源包含边缘云防火墙模块与边缘云入侵检测模块,确保边缘云平台的安全运行,信息安全部分在本文不作为研究重点,故在此不进行展开。

      \begin{figure}
        \centering 
        \makebox[\linewidth]{\includegraphics[width=1.0\linewidth]{fig/chap2/safe_resource.png}}
        % \caption*{图片注释解释说明}
        \caption{CT2ROP场景安全与防护资源分析}
        \label{fig:safety_model}
      \end{figure} 



      \begin{figure}
        \centering 
        \makebox[\linewidth]{\includegraphics[width=1.0\linewidth]{fig/chap2/resource_sim.png}}
        % \caption*{图片注释解释说明}
        \caption{CT2ROP场景资源仿真分析}
        \label{fig:resource_sim}
      \end{figure} 


      资源层的仿真分析如图~\ref{fig:resource_sim}所示。逻辑仿真具体内容类似资源与服务层不具体赘述;MC仿真形式与服务层相同,用于验证资源架构是否能否满足服务规范中对应指标的需求,如实时交通孪生信息提供服务中的纵向误差指标;场景级参数仿真通过将被验证过符合服务标准的资源指标统一输入MATLAB为中介搭建起的微观交通场景仿真中去,从而验证战略能力的各项指标是否能同时被满足。参数来源方面,车路云相关参数来自资源层,而环境定义相关参数来自战略层的各阶段环境参数定义,用以生成符合要求的交通仿真场景,其中参数包含:交通流量(车辆类型与流量情况)、运输任务起止地、道路条件(道路类型与道路几何等)等。通过战略分析阶段对于运输场景的分析确定核心验证场景为江苏、山东等液罐车运量最大事故最为高发省份的高速公路与一级公路,涉及高速路上场景的巡航、汇入、汇出与通过路口场景,以及在一级公路路口通过路口左转弯的场景(对于大型车辆存在右转必停的规定,故并不包含在验证场景中)。在\ref{sub:cloud_decision_sim}节中将详细展示本部分从沿沈海高速盐城至连云港段行驶的液罐车在不同始末点初始化与交通流量、交通流特征的情况下的仿真结果。


\section{本章总结}

本章围绕车云协同液罐车防侧翻场景任务体系架构设计展开,通过理论方法构建与场景化实践验证,为后续车云协同决策规划与控制算法开发奠定了体系基础,核心总结如下:

(1)提出适配 IVCPS 特性的 So-MBSoSE 建模方法。针对 IVCPS“成员独立运行、整体涌现、动态演进” 的体系本质,以及多场景并行交互的核心特点,融合 ABM 自底向上涌现式建模与 EBM 自顶向下指向式建模思想,设计了 “战略 - 运行 - 服务 - 资源” 四阶段架构流程,适配从低渗透率到全智能网联的不同演进阶段,通过逻辑仿真、MC 仿真与场景参数仿真实现架构闭环验证,解决了传统系统工程方法难以适配 IVCPS 开放性与演进性的问题。

(2)完成 CT2ROP 场景任务体系架构全流程设计。以危化品运输侧翻风险防控为核心使命,明确了场景战略动机与演进阶段,划分了智慧危化品运输指挥、路侧智能设施、云控平台、智能网联液罐车四大运行执行者,构建了涵盖共性服务与场景专有服务的服务体系,细化了安全防护与核心软件资源配置,通过 MC 仿真验证了应急响应时间$\le \SI{100}{ms}$ 等关键指标,形成 “云端长时域预测 - 车端高频响应” 的分层规控架构雏形。

(3)明确核心支撑与实践导向。架构设计过程中厘清了液罐车侧翻的根源性、诱发性、状态性与后果性风险,明确了态势感知、风险预测、稳定性控制、鲁棒运营四大核心能力需求,界定了高速公路巡航、汇入汇出、路口转弯等核心验证场景,为第三章车云协同决策规划算法、第四章车液耦合建模与控制算法的开发提供了明确的需求边界、指标约束与场景支撑。