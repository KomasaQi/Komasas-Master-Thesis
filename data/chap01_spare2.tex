% !TeX root = ../车云协同液罐车防侧翻行驶规划与控制研究.tex

\chapter{引言}
\label{chap:introduction}

\section{研究背景及意义}
\label{sec:background_and_significance}

  我国危化品运输需求持续增长,2025年危化品运输总额已突破3万亿元,其中利用液罐车通过公路运输的超过\SI{80}{\percent}\cite{qianyuyan2023},液罐车在危化品公路运输中占据重要地位。据统计,截至2025年,我国危化品运输车辆总数超过36万辆,其中液罐车数量接近19万辆,占比高达\SI{52.8}{\percent}\cite{zhongguowuliulianhehui2022}。然而,液罐车在运输过程中存在较高的安全风险。2011年至2025年间,我国共发生超过1万5千起危化品罐车运输重大事故,其中\SI{78.7}{\percent}的事故发生在液罐车行驶过程中,\SI{62.7}{\percent}的事故是由车辆侧翻引起\cite{liuhuipling2023, chengshuo2023, lijian2014,chenxiaoyong2023, wangyang2023}。液罐车侧翻事故在货车严重伤亡事故中占比超过\SI{31}{\percent},液罐车单方面事故中侧翻占比接近\SI{70}{\percent}\cite{authority_tank_truck2022},可见液罐车侧倾稳定性差易于侧翻的特性。

  频发的液罐车侧翻事故后果严重。如图~\ref{fig:severe_results}所示,2020年6月13日,浙江温岭G15高速,一辆满载25.36吨液化石油气的液罐车因超速侧翻爆炸,导致20人死亡,175人受伤,数十间房屋倒塌,200余间成危楼\cite{wenling_accident2022};2022年四川江油县,一辆液罐车因超速失控侧翻造成8死20伤;山东威海,一辆液罐车路口躲避变道车辆侧翻致多人遇难;2023年广西百色,液罐车在高架桥转弯时因车速过快侧翻爆炸多人遇难;北京东六环,液罐车避让施工侧翻爆炸致多人死亡;湖南邵阳,液罐车下坡急转向时因车速过快侧翻40顿柴油泄漏……因侧翻车辆装载液化石油/天然气、汽柴油、甲醇等易燃易爆品的情况占\SI{86}{\percent}以上\cite{qianyuyan2023},因此事故常伴随危化品液体泄漏,常造成环境污染、燃烧、爆炸等严重次生灾害,酿成重大人员伤亡和财产损失\cite{liuhuipling2023, chengshuo2023, lijian2014,chenxiaoyong2023, wangyang2023,authority_tank_truck2022,wenling_accident2022, baijinghua2023}。故解决液罐车运输过程中的侧翻问题迫在眉睫。


  \begin{figure}
    \centering
    \includegraphics[width=0.9\linewidth]{fig/chap1/severe_results.jpg}
    % \caption*{图片注释解释说明}
    \caption{液罐车事故及其严重后果}
    \label{fig:severe_results}
  \end{figure} 

  液罐车侧翻事故原因可总结为三大部分:液罐车辆本身非满载车液耦合侧向稳定性差易侧翻的车辆特性,驾驶员对动静态交通信息(道路结构、临时施工、周车状态与意图等)掌握不足(单车局限)、超速行驶疲劳驾驶等违法行为以及救车控制失误的人因失误。

  (1)\textbf{车辆特性}:液罐车因车辆液体耦合的非满载状态下侧向稳定性差,易发生侧翻。根据现行标准,液罐车必须非满载运行,以避免液体膨胀、蒸汽压力升高引发的泄漏或爆炸,最大允许充液量不得超过\SI{95}{\percent}\cite{biaozhun185642006,biaozhunR00052011}。液罐车在卸货途中通常保持中低充液率,这会导致液体晃动,明显降低车辆的侧倾稳定性。

  (2)\textbf{单车局限}:驾驶员对动静态交通信息的掌握不足,是正常路段事故的主要原因,主要受限于以单车视角对周围环境、车辆状态和意图的感知能力\cite{authority_tank_truck2022}。特殊路段以及他车危险动态使得驾驶员需及时响应,大曲率弯道和坡道等将改变液体晃动与车辆的侧倾特性若不注意易引起事故。

  (3)\textbf{人因失误}:超速、疲劳驾驶等违法行为,以及紧急工况下由于驾驶员不够熟练而造成的救车失误等由于人因造成的本可以避免的因素而加剧了危险情况下的侧翻风险。

  根据上述分析,如何兼顾车辆特性、单车局限和人因失误三方因素来防止其发生侧翻事故,对提升危化品运输的安全性和可靠性具有重要意义。为增强问题到技术路径的对应关系,可将上述三类致因进一步映射为三类技术抓手,从而形成“风险—机理—方法”的闭环链条:
  \begin{itemize}
    \item \textbf{对应车辆特性(内因)}:需建立能够刻画车液耦合关键模态的动力学模型,并将侧翻约束显式融入规划与控制。
    \item \textbf{对应单车局限(外因信息不足)}:需利用路侧/云端的超视距与全局感知能力,获取道路结构、施工、周车状态与意图等信息,并用于风险推演与预防性决策。
    \item \textbf{对应人因失误(交互与操纵风险)}:需通过自动驾驶与车云协同辅助决策降低超速、疲劳与紧急工况救车失误的发生概率,并提供可验证的安全兜底机制。
  \end{itemize}
  现有的解决方案通常针对上述因素中某一方面开展,但智能网联汽车(Intelligent and Connected Vehicles, ICV)与智能交通系统(Intelligent Transportation Systems, ITS)的快速发展使得液罐车侧翻问题有了更多可以兼顾上述因素的解决方案。

  首先,自动驾驶技术飞速发展,商用车自动驾驶取得了显著成果\cite{hasiri2024}。2024年卡尔动力的2-6车混合无人化半挂车队已完成超过820万公里250万吨货物的无事故运输\cite{kargobot2024};小马智卡L4级自动驾驶卡车大规模商用即将展开\cite{ponyai2024}等标志着商用车自动驾驶技术已经取得长足进步,未来以大宗货物运输为目的的大部分商用货车有望被自动驾驶取代。自动驾驶车辆在不会超速、疲劳驾驶等违反交通法规的前提下可以实现超越人类的感知能力与车辆操纵,为车辆运行安全提供了有效保障。目前小马智卡、卡尔动力、主线科技等自驾重卡公司已获北京市首批智能网联重卡编队路测许可,可在京津冀区域测试道路开展高速自驾货运测试。

  其次,在未来的智慧交通系统中,过去由每辆车组成的交通系统演进为基于智能网联车辆(及存量车、其他交通参与者等)、路侧单元(感知单元、计算单元、通信单元等)、云控基础设施(三层云架构)、相关支撑平台(定位平台、地图平台、交管平台、急救平台)等为一体的车路云一体化融合控制系统,简称云控系统\cite{cuimingyang2022}。云控系统是车、路、云、网、图等成员系统形成的认同型体系\cite{OMG2022,bondavalli2016cyber},是典型的信息物理系统,包含物理层、信息层和应用层。云控系统利用信息实体对物理实体进行孪生并用于控制物理实体,实现融合感知与推演、协同决策规划与控制,形成系统虚实结合的闭环运行,实现交通安全、高效、经济等性能的综合提升\cite{likeqiang2020cloudcontrol}。目前全国已有20个主要城市开展车路云一体化试点工作,云控系统的建立是未来我国交通发展的趋势\cite{gongxinbu2024pilots}。

  \begin{figure}
    \centering
    \includegraphics[width=0.9\linewidth]{fig/chap1/cloud_platform.jpg}
    % \caption*{图片注释解释说明}
    \caption{车路云一体化融合控制系统架构\cite{icv2023whiteroadcloud}}
    \label{fig:cloud_platform}
  \end{figure} 

  需要说明的是,本文所称“车云协同”强调\textbf{云端低频长时域预防性决策与规划}与\textbf{车端高频短时域规划控制兜底}的分层互补;所称“低渗透率”指接入云控服务(具备V2X/云控通信能力)的车辆在交通流中的占比不高的运行阶段。在该阶段,周车行为与意图具有更强的不确定性,且通信时延波动对协同闭环可用性具有直接影响。车云协同并非是与单车智能互斥的两种方案,而是在其基础上辅助单车做更好的决策。其大面积应用得益于其相对于人工驾驶与单车自动驾驶的三条主要优势:全局感知、强大算力、统一调度\cite{likeqiang2020principles}。

  (1)\textbf{全局感知}:精准的全局感知与长时域历史信息使得所有联网车辆可以获得比任何人类驾驶员或单车自动驾驶感知系统更全面的超视距、被遮挡的视野。利用这些感知信息可以为人类驾驶员提供预警性服务,对自动驾驶车辆提供更精准的周车信息与历史数据。

  (2)\textbf{强大算力}:边缘云强大的实时调度并行算力支持一些原本在车端无法实时运行的大计算量高复杂度算法,作为车端短时域规划控制的补充,辅助车端控制器或驾驶员做出更好的决策。

  (3)\textbf{统一调度}:以云为中心的统一调度便于实现多车间的协同优化控制,避免了单车驾驶可能陷入的“零和博弈”或“囚徒困境”。同时区域云和中心云基于交通大数据的分析可以用于对交通层面的整体优化,实现信号灯等交通基础设施的适应性调整。

  综上所述,液罐车侧翻风险同时受车辆-液体耦合动力学(内因)、道路与交通环境(外因)以及驾驶/交互行为(人因)共同驱动,且在危险工况下呈现强不确定与强耦合特征。为使后续研究脉络更清晰,本章首先综述车液耦合机理与液罐车动力学建模研究进展;其次总结液罐车主被动防侧翻措施及其局限;然后从单车自动驾驶决策规划出发,梳理云控预测性决策与规划的关键组织形态、通信约束与可落地问题;最后回到体系工程视角,总结车云协同体系架构设计方法,并在此基础上归纳现有研究不足与本文总体研究方案。


\section{国内外研究现状}

  本研究针对上述由于液罐车车液耦合易侧翻、单车对动静态交通信息掌握不全面、驾驶员违规驾驶及操作不当引起的液罐车侧翻问题,基于云控系统提出的车云协同液罐车防侧翻行驶规划与控制架构主要通过四大方面技术解决问题,包括:体系工程级防侧翻机制设计、罐内液体的精准孪生建模、车端防侧翻轨迹跟踪控制、车云协同预测性决策与防侧翻轨迹规划。为更好开展与实现上述技术,主要调研四大部分国内外现状:车液耦合机理与液罐车动力学建模研究现状、液罐车防侧翻主被动安全措施研究现状、车云协同预测性决策与规划研究现状、体系工程与车云协同体系架构建立。

  \subsection{车液耦合机理与液罐车动力学建模研究现状}
    \subsubsection{车液耦合机理研究现状}
      \label{subsub:vehicle_liquid_couple_mechanism}
      液罐车的车--液耦合效应可从两类物理机制理解:\textbf{准静态质心迁移效应}与\textbf{瞬态晃动冲击效应}。前者主要改变车辆在稳态侧向加速度下的等效质心位置与侧倾力矩,从而降低静态侧翻阈值;后者在转向/变道等非稳态激励下引入额外的冲击侧向力与冲击侧倾力矩,导致动态侧翻阈值进一步降低。如图~\ref{fig:vehicle_liquid_couple_mechanism}所示,以常见类椭圆截面罐体为例,准静态条件下液体自由液面随稳态侧向加速度倾斜,液体质心在侧倾平面内发生\textbf{纵向抬升}与\textbf{横向偏移}:质心抬升会增大车--液总质量对侧倾中心的力矩,横向偏移则直接增大液体重力对车辆产生的侧倾力矩\cite{maohaijian2021fsi},二者共同导致液罐车相较于等质量固体载荷车辆具有更低的静态侧翻阈值。

      在瞬态晃动冲击方面,若将罐内液体等效替换为同体积同质量固体,一般随载荷增加车辆质心升高,侧翻阈值将近似呈单调降低趋势;然而液罐车的侧翻阈值随充液率变化往往呈现\textbf{先快速降低后缓慢回升}的非单调特性,表明晃动冲击与自由液面长度等因素在中等充液率下会放大危险性。大量研究指出\SIrange{50}{70}{\percent}充液率常对应较高风险工况。茅海剑等\cite{maohaijian2021fsi}通过CFD分析不同密度液体晃动,指出高密度液体阻尼更大、晃动衰减更快,并发现圆形罐体侧翻阈值最低的充液率约为\SI{60}{\percent},此时侧向晃动瞬态冲击可超出平均水平\SIrange{40}{50}{\percent}。王小润等\cite{wangxiaorun2022lateralacc}结合准静态分析、CFD与实车试验研究椭圆截面液体冲击机理,指出充液率低于\SI{70}{\percent}时侧向力波动更显著,而高于\SI{80}{\percent}后波动明显受抑。于志新等\cite{yuzhixin2018mpc}基于势流理论的圆形罐体模型表明,随充液率上升惯性分力增加而晃动分力先增后减,二者合力在\SI{50}{\percent}附近达到最大。徐晓美等\cite{xuxiaomei2020lanechange}基于TruckSim--Simulink联合仿真,以单摆模型近似液体晃动,比较\SI{25}{\percent}、\SI{50}{\percent}、\SI{75}{\percent}充液率在单移线、双移线与蛇形工况下的侧倾响应,发现\SI{50}{\percent}充液率与双移线工况分别对应最不利充液率与最不利工况。李杰等\cite{lijie2020carliquid}的CFD结果亦显示晃动力幅值在\SI{60}{\percent}附近达到更不利水平,支持“中等充液率更危险”的共识。

      \begin{figure}
        \centering
        \includegraphics[width=1.0\linewidth]{fig/chap1/two_factors.png}
        \caption{车液耦合机理说明\cite{maohaijian2021fsi,huangyang2017impact,wangqiongyao2017sloshing}}
        \label{fig:vehicle_liquid_couple_mechanism}
      \end{figure}

      除阈值变化规律外,研究还从“晃动强度由何决定”角度给出更细的机理解释。陈益苞等\cite{chenyibao2016crosssection}基于准静态分析指出,侧倾特性主要由液体质心高度与侧倾平面内自由液面长度决定:质心越高,侧倾力矩均值越大;自由液面越长,晃动越剧烈且高阶分量更丰富。王琼瑶\cite{wangqiongyao2017sloshing}通过CFD对比含/不含弹性气袋的部分充液罐体,指出侧向晃动频响中占主导的频率成分取决于激励加速度幅值大小,表现为激励频率或液体固有频率的主导切换。

      \begin{figure}
        \centering
        \includegraphics[width=0.65\linewidth]{fig/chap1/coupling_dynamics.png}
        \caption{车液耦合动力学特征}
        \label{fig:coupling_dynamics}
      \end{figure}

      综合上述研究(见图~\ref{fig:coupling_dynamics}),车--液耦合对车辆侧向动力学的主要影响可概括为:\textbf{(i) 响应幅值增大}(准静态质心迁移抬升等效侧倾力矩),\textbf{(ii) 超调与峰值冲击显著}(瞬态晃动冲击引入额外侧向力/侧倾力矩),以及\textbf{(iii) 阻尼偏小导致振荡持续}(流体等效阻尼有限且能量耗散慢)。这些特性共同解释了液罐车在中等充液率与强横向激励工况下更易出现侧倾失稳的现象。因此,后续建模方法的选择需在‘可解释的关键模态表达’与‘实时性’之间权衡。

    \subsubsection{液体晃动建模研究现状}
      \label{subsub:liquid_shaking_modeling}

      液体晃动建模最早主要服务于飞行器燃料箱(飞机油箱、火箭燃料箱等)、船舶配重水舱以及车辆油箱等场景,近年来逐步扩展至液罐车晃动与侧倾稳定性问题。针对晃动建模,学术界形成了多类方法体系。贾心红等\cite{jiaxinhong2022review}对相关方法进行了系统归纳,可概括为四类:\textbf{准静态模型法}、\textbf{机械等效力学模型法}、\textbf{液体晃动动力学(流体)模型法}与\textbf{实验研究法}(见图~\ref{fig:liquid_modeling_methods})。


      \begin{figure}
        \centering
        \subcaptionbox{准静态模型法\cite{huangyang2017impact}\label{fig:liquid_model_qs}}
          {\includegraphics[width=0.40\linewidth]{fig/chap1/quasi_static.png}}
        \subcaptionbox{机械等效力学模型法\cite{yuzhixin2019optimalcontrol,zhaoweiqiang2018equivalent,zhaoweiqiang2019semitrailer}\label{fig:liquid_model_equivalent_model}}
          {\includegraphics[width=0.45\linewidth]{fig/chap1/equivalent_model.png}}
        \vspace{0.2cm}
        \subcaptionbox{液体晃动动力学模型法\cite{maohaijian2021fsi,yuzhixin2018mpc}\label{fig:liquid_model_cfd}}
          {\includegraphics[width=0.35\linewidth]{fig/chap1/fluid_dynamic_model.png}}
        \subcaptionbox{实验研究法\cite{wangxiaorun2022lateralacc,wanying2020antirolover}\label{fig:liquid_model_experiment}}
          {\includegraphics[width=0.45\linewidth]{fig/chap1/experiment_method.png}}
        \caption{液体晃动建模方法}
        \label{fig:liquid_modeling_methods}
      \end{figure}


      \paragraph{(1) 准静态模型法(QS)}
      准静态模型(Quasi-Static Model, QS)主要描述稳态工况下自由液面与液体质心位置,忽略动态晃动过程,通常仅利用截面几何与静力平衡关系进行解析/半解析表达。该方法在稳态侧向加速度下对平均晃动力/侧倾力矩的估计具有一定有效性,但在变道、蛇形等瞬态工况下,由于晃动冲击与相位滞后显著,QS模型误差会明显增大。黄洋等\cite{huangyang2017impact}对比QS与CFD分析瞬态冲击,指出峰值冲击可显著高于平均值,并定义瞬时液体冲击影响因子以表征瞬态冲击导致侧翻阈值相对QS结果的折减;该因子在\SI{0}{\percent}与\SI{100}{\percent}充液率时为0,并在\SI{50}{\percent}附近达到最大(\SI{22.61}{\percent})。何烈云等\cite{helieyun2020quasistatic}建立椭圆截面罐体的准静态等效模型,分析侧倾稳定性与截面形状、液体装载率等因素的关系,进一步说明QS方法在稳态设计与参数敏感性分析中的价值。

      \paragraph{(2) 机械等效力学模型法}
      机械等效模型以“输入输出等效”为原则,将液体系统视为黑箱,通过构造单摆、弹簧--质量--阻尼、多摆或椭圆规摆等机械系统来逼近液体晃动所产生的等效力/力矩。其优势在于\textbf{计算实时性好、易于嵌入控制与规划},并可在一定范围内准确描述一阶/低阶晃动;局限在于模型能表达的模态受结构形式限制,常难以覆盖水跃、飞溅等强非线性现象,且更高阶模态需要更复杂的等效结构。Salem\cite{salem2000rollover}较早提出椭圆规摆(Trammel Pendulum, TP)用于液罐车液体建模,并指出其对椭圆罐体在较大侧向加速度范围内往往优于单摆模型。杨秀建等\cite{yangxiujian2018multimass}与吴相稷\cite{wuxiangji2018sloshingmodel}提出多质量椭圆规摆模型,并对比证明多质量结构可提升多工况下等效力拟合精度。戚笑景等\cite{Komasa2023observing}提出广义摆模型(Generalized Pendulum Model, GPM)概念,其中双质量椭圆规摆(DMTP)与组合双摆(TPSP)等可将二维晃动力特性的等效精度显著提升,在部分工况下可逼近CFD结果。

      \paragraph{(3) 液体晃动动力学模型法}
      该类方法直接基于流体动力学微分方程描述液体流动,可在连续流体域或离散网格(有限体积/有限元等)内求解晃动响应,典型包括线性晃动理论与CFD数值方法。线性理论通常适用于简单几何与小幅晃动,推导复杂但有助于机理分析;CFD方法可捕捉强非线性、分离、飞溅等高阶现象,精度优势明显,且借助Fluent、StarCCM+等软件可较方便建模,但计算开销巨大,通常更适合作为高精度基准或离线标定工具,而不适用于实时控制系统的在线求解。

      \paragraph{(4) 实验研究法}
      实验研究可直观获取真实晃动行为与车辆响应,但受安全性与成本限制,涉及实车与真实危化品的高风险试验较少;多数工作基于相似性原理采用缩比液罐与台架/缩比实车平台验证模型与控制策略。王小润等\cite{wangxiaorun2022lateralacc}在室内场地借助台架辅助开展实车试验,并在约\SI{0.1}{g}侧向激励下与CFD结果对比验证线性晃动条件下CFD可靠性;万滢\cite{wanying2020antirolover}构建缩比液罐车模型并开展缩比实车实验,将测得的加速度作为CFD输入并与实验数据对比,验证联合仿真平台的一致性。

      综合比较四类方法:机械等效模型在能够表达低阶晃动的前提下具备较高实时性,因而更适用于车辆规划与控制;但其表达能力依赖等效结构设计,现有模型多聚焦于横向或纵向截面内的二维晃动,难以充分反映对车辆动力学同样重要的输出模态(例如横摆/俯仰方向的等效力矩),且往往忽略纵向加减速与横纵向流动耦合对横向晃动的影响,这为后续在复杂交通工况下的精确预测与鲁棒控制带来挑战。

    \subsubsection{液罐车动力学建模研究现状}
      \label{subsub:vehicle_dynamics}
      在规划与控制中引入液罐车动力学约束的前提是建立可用于计算的车--液耦合动力学模型。考虑工程应用的实时性需求,常见建模路线是以\textbf{简化车辆横向/横摆模型}为骨架,并与\textbf{液体晃动模型}(QS/等效力学/流体模型等)耦合形成低阶可计算模型。对整体式液罐车而言,车辆部分常采用自由度(Degree of Freedom, DOF)自行车模型或3DOF模型(纵向、侧向、横摆)与液体模型耦合;为刻画侧倾风险,也有研究引入包含侧倾自由度的4DOF模型,或以侧倾、横摆、侧向为核心的低阶模型与液体模型结合\cite{yudi2019rollstability}。任园园等\cite{renyuanyuan2020precisemodel}采用椭圆规摆近似液体并与3DOF车辆模型耦合,推导得到4DOF液罐车动力学模型,用于数值仿真分析液罐车相较固体载货车辆的运动学特性差异;Matthew等\cite{aquaro1999fem}基于有限元思想离散罐体质量,使用梁单元与单摆分别描述车体与液体;万滢\cite{wanying2020antirolover}采用椭圆规摆进行液体建模,并进一步用等效单摆分别与3DOF与4DOF车辆模型耦合,为LQR抑晃与防侧翻控制提供模型依据。

      对于半挂液罐车,由于牵引车与挂车通过铰接耦合,完整动力学模型可包含牵引车与挂车的平面运动(纵向、侧向、横摆)以及侧倾、俯仰等多自由度,维数较高;在现有研究中通常按研究目标进行简化。常见的车辆简化模型包括:牵引车纵向、侧向、横摆加挂车横摆的4DOF模型(如制动工况分析\cite{heren2016braking}),以及牵引车侧向、横摆、侧倾加挂车横摆、侧倾的5DOF模型\cite{zongchangfu2014identification,niezhigen2015heavyduty}等。于志新等\cite{yuzhixin2018mpc,yuzhixin2019optimalcontrol}基于势流理论建立液体晃动方程,并与5DOF半挂车动力学模型耦合,构建了6DOF半挂液罐车模型,体现了“车辆简化模型 + 可计算液体模型”的主流路线。

      获得模型结构后,\textbf{参数辨识}是决定模型可用性的关键环节。常用做法是基于最优化搜索(遗传算法、粒子群等)最小化模型输出与高精度仿真/实验输出之间的误差以估计参数。万滢\cite{wanying2020antirolover}利用遗传算法对等效单摆与椭圆规摆模型进行参数拟合,使其匹配\SI{1}{m/s^2}侧向激励下CFD的侧向力输出;Yu等\cite{yudi2016nonfull}对比不同突变率与种群规模配置下遗传算法的辨识性能,指出当种群规模较小(如小于80)时突变率过大或过小均可能导致最优性下降。对于车辆部分参数,宗长富\cite{zongchangfu2014identification}与聂枝根等\cite{niezhigen2015heavyduty}采用“双模型 + 遗传算法”的方式进行多工况辨识,并建立参数插值表以适配运行时工况变化。

      总体而言,半挂液罐车的工程化建模常以5DOF左右的半挂车动力学简化模型为基础,叠加可实时计算的液体晃动模型(其中线性化单摆/椭圆规摆等机械等效模型最常见),在精度与实时性之间取得折中;但现有模型多仍存在横纵向耦合不足、输出模态受限以及在复杂交通工况下泛化能力有限等问题,为后续面向预测性规划与鲁棒控制的模型升级提供了明确方向。

  \subsection{液罐车防侧翻主被动安全措施研究现状}
    通过液罐车自身的改进解决其侧倾稳定性差问题的思路有两大类,一类是以改进车辆自身性能的被动措施;另一类是从控制入手的主动安全措施,下面对两种思路分别介绍。
    \subsubsection{液罐车防侧翻的被动安全措施研究现状}
    \label{subsub:passive_safty}

      被动防侧翻措施主要从\textbf{车辆结构与罐体内部构型}入手,通过改变液体晃动模式、降低液体质心迁移幅度、改善罐体受力路径等方式提升侧倾稳定性。现有研究与工程实践中,典型被动措施如图~\ref{fig:passive_methods}所示,包括:\textbf{罐体形状(尤其截面形状)优化}、\textbf{防浪板/防波板设计}以及\textbf{固定自由液面或抑制晃动的装置(如弹性气袋)}等。

      \paragraph{(1) 罐体形状优化}
      罐体截面形状直接影响液体质心位置及其在横向激励下的迁移规律,从而影响侧倾稳定性。陈益苞等\cite{chenyibao2016crosssection}建立了相同面积的圆形截面、两种改进椭圆形截面以及三种锥形截面的QS模型,对比了不同充液比下多种侧倾稳定性指标,验证其提出的部分锥形截面在提升侧倾稳定性方面具有优势。瞿绘军等\cite{quhuijun2017calculation}依据GB28373-2012《N类和O类罐式车辆侧倾稳定性》\cite{gb28373_2012}的计算方法,并以GB7258-2017《机动车运行安全技术要求》\cite{gb7258_2017}的稳定性判定条件为依据,对某型号液罐车进行理论计算,指出变截面罐体相较非变截面罐体可降低质心高度,且在设计许可范围内变截面高度越大整车重心越低。于迪等\cite{yudi2016nonfull}建立$n$阶弹簧--质量等效模型描述罐体线性横向振动模态,研究不同椭圆率罐体后指出椭圆率$a_{t}/b_{t}=1.5$时侧向冲击力与侧倾力矩较小。进一步地,Di Yu等\cite{yudi2019rollstability}利用遗传算法优化椭圆规摆等效模型参数以最小化侧倾力矩,发现当$a_{t}=\SIrange{1.6}{1.7}{m}$、$b_{t}=\SIrange{1.15}{1.25}{m}$(即$a_{t}/b_{t}=\SIrange{1.28}{1.478}{}$)时稳定性较优;李显生\cite{lixiansheng2015ga}亦采用遗传算法优化椭圆率并以液体质心高度与侧倾力矩为目标,得到最优椭圆率约为1.367。

      \paragraph{(2) 防浪板/防波板设计}
      通过设置防浪板抑制液体横向流动是工程上常见且有效的被动手段。Guillermo等\cite{moreno2022davies}采用质点簇--杆件等效模型研究液体质心迁移与车辆侧倾稳定性之间的关系,提出新的稳定性指标,并给出“抑制液体侧向流动的装置可提升侧倾稳定性”的结论。刘奎等\cite{liukui2009braking,liukui2009steering}通过CFD仿真分析制动与转向工况,指出纵向防波板有效面积大于横截面\SI{40}{\percent}时可有效改善受力,并给出按容积间隔布置纵向防波板的工程建议;同时,纵向防波板的设置也与现行安全规程的设计要求相一致。相比之下,横向防波板由于缺乏统一规定且制造成本较高,工程应用中较少采用。

      \paragraph{(3) 固定自由液面/结构改进与信息辅助}
      除上述两类主流方向外,研究还探索了通过弹性气袋等装置固定自由液面以抑制晃动\cite{wangqiongyao2017sloshing}、通过罐体结构改进优化受力\cite{fangliang2022fe}、以及利用地理信息系统提供道路结构预警信息\cite{zhangweihua2019gist}等思路,以从液体晃动、结构强度或驾驶辅助层面降低翻车风险。

      \begin{figure}
        \centering
        \subcaptionbox{改进液罐截面\cite{chenyibao2016crosssection}\label{fig:cross_sec_improve}}
          {\includegraphics[width=0.31\linewidth]{fig/chap1/tank_shape.png}}
        \subcaptionbox{改进防浪板形状\cite{hanyoubin2019vof}\label{fig:wave_board_improve}}
          {\includegraphics[width=0.31\linewidth]{fig/chap1/wave_proof.png}}
        \subcaptionbox{增加弹性气袋\cite{wangqiongyao2017sloshing}\label{fig:elastic_bag}}
          {\includegraphics[width=0.31\linewidth]{fig/chap1/elastic_bag.png}}
        \caption{液罐车防侧翻被动安全措施}
        \label{fig:passive_methods}
      \end{figure}

      \paragraph{(4) 小结:被动措施的边界与向主动/协同的自然过渡}
      综上,被动安全措施围绕罐体形状、防浪结构与自由液面抑制等方面已形成较为系统的研究与工程应用。受限于整车成本、轻量化、检修/清洗便利性等约束,罐体内部难以引入过于复杂或维护成本高的构件,使得多数“可产业化”的方案在边界工况下仍存在剩余风险。现实事故中,液罐车侧翻往往还与道路几何突变(如小半径弯道、纵坡叠加)、驾驶员对风险感知滞后、对周车动态预测不足或违规操作等外部因素耦合有关。因而,在车路云一体化推进的大背景下,若能在车云协同架构下将\textbf{主动风险预测、预防性轨迹规划与控制干预}结合,有望在进入危险状态之前降低侧翻风险,并与被动措施形成互补的分层安全体系。
      
    \subsubsection{液罐车防侧翻主动安全措施研究现状}
      \label{subsub:active_safty}

      面向液罐车的主动防侧翻通常可归纳为两条互补路径:\textbf{防侧翻控制}与\textbf{防侧翻运动规划}。前者侧重在已进入高风险状态时通过执行器施加控制作用抑制侧倾失稳(更偏“救车”);后者侧重在决策与轨迹层提前约束侧向激励与姿态响应,从源头降低进入危险状态的概率(更偏“预防”)。考虑本文关注的半挂式液罐车,以下从侧翻风险衡量指标出发,分别综述防侧翻控制与防侧翻运动规划研究,并补充半挂车规划与跟踪的特点与现状。

      \paragraph{(a) 车辆侧翻状态衡量指标}
      车辆侧翻风险的常用指标包括:基于质心高度与轮距的静态稳定因数(Static Stability Factor, SSF)\cite{huston2014ssf}、动态稳定因数(Dynamic Stability Factor, DSF)\cite{jin2007critical,heyuancha0_2010factor}、侧向载荷转移率(Lateral-load Transfer Ratio, LTR)\cite{zhang2016pulsed,zhao2019hinfty}及其变体\cite{jin2021heavytruck}、多参数联合指标\cite{jin2016tripped,jin2019triaxle,imine2015switched}、侧翻时间(Time to Rollover, TTR)\cite{wanying2020antirolover,zeng2017bpnn}、零力矩点(Zero Moment Point, ZMP)\cite{jinliqiang2017yujing}等。总体而言,SSF难以反映动态侧翻过程;DSF涉及参数多且对工况敏感;LTR物理意义直观但往往需借助悬架力等量进行等效估计;TTR更适用于预警而非常规闭环控制;ZMP常用于机器人稳定性分析,在车辆侧翻控制中的使用相对有限。上述指标的选择直接影响控制器/规划器对“风险”的定义以及干预触发逻辑。

      \paragraph{(b) 防侧翻控制:执行器手段与控制算法谱系}
      面向车辆侧倾稳定性的控制手段传统上可归纳为四类(见图~\ref{fig:anti_rollover_methods}):主动悬架\cite{xiao2020rollsuspension}、主动转向\cite{shao2023vhrobustmpc}、差动制动\cite{carlson1981two}、分布式驱动\cite{zilin2023integrated},以及多手段组合(如差动制动与主动转向、主动转向与主动悬架等)\cite{xu2011rollover,liang2022multiagent}。
      
      主动悬架通过调节左右悬架支撑力抑制质心外移,但硬件与能耗成本高且在极端工况抑制能力有限,商用车落地受限,目前除少数高端方案(如采埃孚ZF的CDC连续减振阻尼控制)外应用较少。
      主动转向通过调节前轮(和/或后轮)转角改变车辆横向运动状态,其本质可视为“在控制层实施的路径重规划”,在极端工况下具备更强的翻车避免能力,但对硬件(电子助力转向与线控/主动转向能力)依赖更强。
      差动制动通过左右轮制动力不均产生横摆力矩,间接影响侧倾动力学,优势在于可在已普及的电子稳定系统(Electronic Stability Program, ESP)硬件上低成本实施,但其对侧倾动态的作用路径间接且通常偏被动,典型工程应用包括美驰威伯科的RSC防侧翻稳定性控制系统。
      分布式驱动可视为差动制动的扩展,通过在多个车轮施加驱/制动力增强可用横摆力矩,但通常依赖轮毂电机或带动力挂车等结构;其效果同样受制于“横摆力矩$\rightarrow$侧倾响应”的间接性与上限。

      \begin{figure}
        \centering
        \subcaptionbox{主动悬架\label{fig:active_suspension}}
          {\includegraphics[width=0.40\linewidth]{fig/chap1/active_suspension.png}}
        \subcaptionbox{主动转向(前轮和/或后轮)\label{fig:active_steering}}
          {\includegraphics[width=0.40\linewidth]{fig/chap1/active_steering.png}}
        \vspace{0.2cm}
        \subcaptionbox{差动制动\label{fig:diff_braking}}
          {\includegraphics[width=0.40\linewidth]{fig/chap1/diff_braking.png}}
        \subcaptionbox{分布式驱动\label{fig:distributed_driving}}
          {\includegraphics[width=0.40\linewidth]{fig/chap1/dist_driving.png}}
        \caption{现有车辆防侧翻控制方法}
        \label{fig:anti_rollover_methods}
      \end{figure}

      由于现有液罐车防侧翻控制研究多以人工驾驶场景下的辅助驾驶为对象,受限于车载硬件条件,工程上更常采用差动制动,少量工作将差动制动与主动转向结合。在控制算法层面,常见方法包括比例积分微分控制(Proportional--Integral--Derivative, PID)及模糊PID、线性二次型调节器(Linear Quadratic Regulator, LQR)、模型预测控制(Model Predictive Control, MPC)、滑模控制(Sliding Mode Control, SMC)、无模型自适应控制(Model-free Adaptive Control, MFAC)\cite{huyun2019mfac,luoyugong2019sbw,tangzeyue2023mpcmfac,fengzengxi2021mfacpso}、$H_\infty$控制\cite{zhao2019hinfty}等。

      具体而言,赵伟强、封冉等\cite{zhaoweiqiang2018equivalent}建立等效单摆与3DOF液罐车辆耦合模型,采用PID控制差动制动实现防侧翻;李杰等\cite{lijie2020carliquid}通过在Fluent中加入自定义动量源项实现车--液耦合仿真,验证模糊PID差动制动策略;于志新、程新新等\cite{yuzhixin2018mpc}基于势流理论建立液体晃动模型并与3DOF车辆模型耦合,采用MPC同时控制差动制动与主动转向;赵伟强、凌锦鹏等\cite{zhaoweiqiang2019semitrailer}将等效单摆与6DOF半挂液罐车耦合并采用LQR进行差动制动侧倾控制;于志新等\cite{yuzhixin2019optimalcontrol}同样将势流液体模型与6DOF半挂车结合并采用LQR控制;Sun等\cite{Reviewer2_SunWencai2022}将单摆与3DOF车辆结合并基于LQR误差模型实现侧倾稳定性控制;万滢\cite{wanying2020antirolover}以TTR为指标,基于等效单摆与3DOF车辆模型构建横摆--抑晃--防侧翻最优控制系统并通过LQR控制差动制动;魏星\cite{weixing2023_antirollover}提出半挂式液罐车防侧翻与路径跟踪协调控制,将主动防倾杆、差动制动与轨迹跟踪通过状态机切换实现多模态控制;郑雪莲等\cite{zhengxuelian2019mfacpatent}提出基于FFDL-MFAC的侧翻稳定性控制,无需精确动力学模型即可通过附加差动制动控制横摆角速度,适用于不同罐体形状且无需事先标定;Li等\cite{MFAC_LiXiansheng2021}为差动制动与主动转向设计MFAC,并指出主动转向在稳定时间与超调方面相对差动制动具有显著优势。该差异与如下事实一致:横摆力矩对侧倾动力学的影响通常间接且受限\cite{liutianfeng2024heterogeneous},而主动转向通过改变车辆横向运动状态具有更强的翻车避免潜力。

      总体来看,上述控制方法能够在一定程度上提升液罐车侧倾稳定性,但多聚焦于人工驾驶条件下的\textbf{紧急救车}或\textbf{事后抑制},难以覆盖自动驾驶/协同驾驶所要求的常态化运行与预防性安全保障。因此,若将控制与轨迹层的防侧翻规划相结合,有望在进入危险区域之前就降低侧翻风险,从而显著提升系统有效性。

      \paragraph{(c) 防侧翻运动规划:在规划层显式约束侧倾动力学与风险指标}
      防侧翻运动规划通常在通用轨迹规划框架上引入车辆侧倾动力学约束或风险代价(如LTR/TTR等),以限制侧向激励峰值及其变化率,从源头降低失稳概率。通用轨迹规划方法可概括为:基于搜索、基于采样、基于几何参数化、基于优化、基于学习等五类。

      基于搜索的方法常见有Dijkstra\cite{bacha2008odin}、A*\cite{likhachev2009planning}、Hybrid A*\cite{kurzer2016hybrida}、D*\cite{stentz1994efficient}、分阶段动态规划(Dynamic Programming, DP)\cite{dong2023ecocruising}以及结构化道路离散搜索的Lattice Planner\cite{werling2010frenet}等。这类方法在离散空间中具有完备性,但由于状态空间离散化,通常难以充分刻画车辆动力学与连续执行器约束。
      基于采样的方法如PRM\cite{kuwata2009realtime}、RRT\cite{karaman2011sampling}、RRT*\cite{klemm2015rrtconnect}、双向RRT*\cite{fan2024birrt}等,同样具有完备性,但随机性使得结果一致性与可重复性相对较弱,常用于非结构化环境。
      基于几何的方法在低速场景常用Dubins\cite{siedentop2015dubins}、正弦Sine曲线\cite{huang2022sineresistance}等;在高速场景常用回旋曲线/欧拉螺线\cite{brezak2014clothoids}、空间坐标多项式\cite{petrov2014overtaking}、贝塞尔曲线\cite{rastelli2014continuouscurvature}、非均匀B样条\cite{phanhuu2021bsplines}、样条曲线\cite{gotte2018splinebased}、PCHIP三阶插值\cite{hegedus2020realtime}等,优势在于曲率连续与控制自由度较高。
      基于优化的方法通过优化控制量或曲线参数生成光滑轨迹,并引入碰撞约束\cite{dong2023ecocruising}、人工势场\cite{liuzhiqiang2017shichang}等代价/约束;例如Rasekhipour等\cite{rasekhipour2017pfmpc}构建三维虚拟势场并采用多约束MPC跟踪;Jin等\cite{jin2021heavytruck}提出基于MPC的重型卡车防侧翻路径规划与控制策略,结合V2X与环境信息、人工势场与动力学限制实现路径规划与轨迹跟踪。
      基于学习的方法通常采用强化学习(Reinforcement Learning, RL)\cite{wu2024rlsurvey,zhang2023intersectionrl,irshayyid2024highwayrl},并与Safe RL理论\cite{zhang2024saferl_tits,gu2024saferl_tpami,garcia2015saferl}结合以提升安全性,但训练高度依赖数据质量与场景覆盖,且在实际驾驶中可能表现为过度保守或不够安全的两端化状态,仍处于快速发展阶段。

      总体而言,上述规划方法各有优势与局限,但\textbf{公开文献中针对液罐车特性(液体晃动耦合、半挂结构、侧倾风险强约束)的防侧翻运动规划方法仍较少},该方向存在明显研究空白。

      \paragraph{(d) 半挂车运动规划与轨迹跟踪:内轮差、折叠约束与牵挂协同}
      本文面向半挂式液罐车。相较于单车,半挂车规划除需满足动力学约束(如防侧翻、防折叠等失稳约束)与执行器约束外,还需处理牵引车与挂车“不同轨”的内轮差效应,并同时满足牵引车与挂车对静态/动态障碍的避障要求。

      现有半挂车规划与跟踪控制大体可分为低速与高速工况。低速工况主要对应自动泊车:规划层基于半挂车运动学模型同时考虑牵引车与挂车避障,采用搜索/采样/优化/强化学习等方法规划牵引车泊车轨迹,控制层通常只需牵引车或挂车一侧跟踪即可实现牵挂整体避障与约束满足。例如王元民等\cite{wangyuanmin2024}将二次规划(Quadratic Programming, QP)与双向拓展快速拓展随机树(Bi-RRT)结合求解初始解,并用偏向性采样平滑实现半挂车避障倒车;陈朝\cite{chenchao2023}将铰接角与碰撞约束与B样条优化结合实现垂直泊入;贾生超\cite{jiasenchao2023}推导前向/后向铰接角稳定条件并与混合$A^\ast$结合实现自动泊车。
      高速工况下的半挂车轨迹规划往往弱化内轮差效应,更侧重先生成轨迹再通过\emph{跟踪控制}使牵引车与挂车尽量跟踪同一条轨迹,或令挂车跟踪目标轨迹:如李斯旭等\cite{lisixu2021}结合多点预瞄与状态轨迹线性化加权牵引车与挂车侧向跟踪误差;曹莉凌等\cite{caoliling2024}利用LQR调节铰接角误差并用PID进一步补偿;张发旺等\cite{zhangfawang2025}提出双策略跟踪,上层用近似动态规划(Approximated Dynamic Programming, ADP)控制牵引车跟踪、下层用IPOPT优化权重以降低挂车跟踪误差;宋广昊\cite{songguanghao2023}提出羊角螺线接贝塞尔曲线的两段式换道轨迹以降低峰值LTR,并通过挂车轴主动转向实现挂车轨迹跟踪;王洪昌\cite{wanghongchang2023}提出基于同伦流形的混合A*/DP规划方法以实现路口离散化后的快速半挂车避障规划。总体来看,低速方案多基于运动学模型,难以直接推广到高速且常依赖离线搜索导致实时性受限;高速方案下牵引车与挂车跟踪目标可能存在冲突,协同处理复杂,尤其在侧倾强约束与液体晃动耦合条件下仍缺乏统一而可落地的规划--控制一体化方案。

  \subsection{车云协同预测性决策与规划研究现状}

    云端预测性行为决策与运动规划是在单车自动驾驶决策与规划的基础上发展而来的,相对于单车方案,云端预测性的决策和规划更加考虑了云端的全局感知、并行算力与协同优化优势,对单车所触及不到的长时域规律进行解析,并给出更具指导性的建议。通常在车云协同的决策与规划限于通信时延等因素采用快慢双闭环的思想,即云端进行低频长时域的预测性决策与规划,而车端负责高频短时域的实时决策规划。

    \subsubsection{自动驾驶决策与规划研究现状}
      \label{subsub:autonomous_decision}

      尽管云端与车端在可用信息量、更新频率与规划颗粒度上存在差异,其决策与规划的基本范式是相通的:在不确定、多主体交互的交通环境中,生成既安全又高效且可执行的行为与轨迹。因此,单车自动驾驶的决策与规划研究为云端预测性决策提供了重要参考。自动驾驶行为决策自始至终试图解决的核心问题只有一个:\textbf{如何处理自车与周围交通参与者之间的交互}。

      自DARPA自动驾驶挑战赛以来,感知--预测--决策--规划--控制的模块化流程被沿用多年\cite{ferguson2008urban,urmson2008boss,leonard2008perception,kammel2008annieway,vonhundelshausen2008tentacles,montemerlo2008junior,thrun2006stanley},但该流程隐藏了一个关键逻辑矛盾:若先对周车进行预测再据此决策,则预测结果默认与自车决策无关;而实际交通中周车行为往往会对自车动作产生反应,预测应当随自车候选决策而变化。由此,传统“先预测后决策”的串行结构在多车交互场景下容易出现保守、迟滞或不稳定等问题。

      \paragraph{(1) 预测--决策一体化与交互建模:模块化路线的主流演进}
      为修正上述矛盾,近年来模块化路线的一个共同趋势是\textbf{预测--决策一体化}:将自车决策作为条件变量,显式或隐式地建模“自车动作$\rightarrow$他车响应”的耦合关系,并在候选决策之间做综合评估。现有研究大体可按交互处理方式划分为显式交互、隐式交互及综合式交互三类。

      \textbf{显式交互}以自车候选策略/意图划分场景,在每个候选行为下预测周车反应,并对不同候选行为的长期收益与风险进行比较,典型代表为多策略决策方法(Multi-Policy Decision Method, MPDM)。Cunningham等\cite{cunningham2015mpdm,galceran2017changepoint}提出MPDM思想,将交互不确定性建模为部分可观测马尔可夫决策过程(Partially Observable Markov Decision Problem, POMDP),并用一组闭环策略/意图近似描述自车与周车的决策空间,从而显著压缩搜索规模。Ding等提出的EPSILON\cite{ding2022epsilon}继承并改进该思想,提出领域特定闭环策略树(Domain-Specific-Policy Tree, DCP-Tree)与条件集中分支(Conditional Focused Branching, CFB)机制\cite{zhang2020guidedbranching},通过限制自车策略展开深度并聚焦意图不确定性高的周车,提升了在线规划的效率;并进一步结合时空语义走廊规划器\cite{ding2019semanticcorridor}用于运动规划。Bae等\cite{bae2023lanechange}结合MPC与生成对抗网络(Generative Adversarial Network, GAN)生成平滑换道轨迹,并引入自适应安全边界与卡尔曼滤波以提升密集交通中的鲁棒性与可用性。

      \textbf{隐式交互}侧重对周车意图/不确定性构造多个可能情景,并在情景集合上评估自车应对策略,典型代表包括防御性规划(Contingency Planning)\cite{salvado2016contingency}与博弈论方法。Huang等\cite{huang2021riskbudget}提出基于风险预算的递归控制算法,通过动态分配风险预算在满足安全约束的同时减少过度保守。Da等\cite{da2022reactivesafety}提出综合反应式安全(Comprehensive Reactive Safty, CRS)框架,强调规划不必在所有未来情形下“绝对无碰撞”,而应能对环境变化做出反应并保证安全,并在此基础上提出反应式迭代线性二次规划(iterative Linear Quadratic Regulator, iLQR)算法。博弈论思路方面,Le Cleach等\cite{li2018gametheoryacc}提出基于无迹卡尔曼滤波的逆动力学博弈用于在线估计其他智能体目标函数,并将其整合到滚动时域博弈规划器中;进一步在\cite{li2020gametheorytits}提出增广拉格朗日快速求解器ALGAMES以求解含非线性约束的多玩家动态博弈问题。Li等\cite{lecleach2020lucidgames}提出面向无信号交叉口的多车博弈框架,并在\cite{lecleach2022algames}提出基于Level-K博弈论的长时域、多步交互决策模型,体现了“交互可解释、决策可推演”的特点。

      \textbf{综合式交互}将显式交互与隐式交互思想融合,并常与风险度量、礼貌性或协同性等机制结合。Tong Li等\cite{li2023marc}提出风险敏感多策略防御性规划(Multipolicy and Risk-Aware Contingency Planning, MARC)框架,集成MPDM与风险感知的防御性规划,并通过场景级动态分支点与LP-iLQR双层迭代优化显著提升复杂交互环境下的决策效率与自然性。此外,Sun等\cite{sun2018courteous}在目标函数中引入“对他车成本影响”的度量以形成更礼貌的驾驶行为;Sheng等\cite{sheng2023cooperationaware}利用时空图卷积网络构建交互式轨迹预测模型,用于评估他车对自车决策的反应,从而增强协同意识。

      \paragraph{(2) 端到端路线:以学习方式实现耦合闭环,但引入可验证性与可控性挑战}
      除模块化路线外,端到端自动驾驶通过学习从环境表征到行为/轨迹的联合映射(例如从栅格/矢量/图结构表征直接生成未来轨迹或控制序列),在训练阶段将预测、决策与规划的耦合显式纳入优化目标,从而在一定程度上缓解传统“先预测后决策”带来的交互一致性问题。常见实现包括基于模仿学习的端到端轨迹生成、基于强化学习的交互策略学习,以及以扩散/生成模型为代表的多模态规划生成等。端到端方法的优势在于能够以统一目标函数刻画交互与不确定性并生成多样化候选解,但其在数据依赖、可解释性与安全可验证性方面仍面临挑战,工程上通常需要与规则约束、安全壳或可行域投影等机制结合。结合本文关注的安全敏感商用车场景,端到端更适合作为云端长时域推演与候选策略生成的工具,而车端仍需在安全约束下完成可验证的实时闭环执行。本文在附录\ref{sub:end2end}中对端到端路线作进一步综述。


      \paragraph{(3) 运动规划:时空解耦与时空联合的两类求解范式}
      在得到行为/意图层决策后,运动规划模块需生成一条可跟踪的三维时空轨迹或二维平面路径。对于高速运动、安全敏感且多主体实时交互的结构化道路任务,通常需要对轨迹进行规划与跟踪。对非学习类方法而言,轨迹规划的差别主要体现在求解维度与约束处理方式:自动驾驶一般考虑车辆在平面内二维运动,故轨迹规划涉及三维时空求解。针对三维时空轨迹规划,现有研究可分为时空解耦规划与时空联合规划两类路线。

      时空解耦思路将三维规划拆解为两个二维规划问题。代表性案例如百度Apollo的EM-Planner\cite{fan2018apollo}采用横纵向解耦:先规划路径,再在ST图中进行碰撞检测,利用动态规划(Dynamic Programming, DP)求解离散空间中的速度曲线,并通过二次规划(Quadratic Planning, QP)进行平滑。该类方法计算效率高,但由于忽略横纵向耦合可能损失最优性。时空联合规划在计算上更昂贵,但不人为限制解空间,尤其对体积较大、可行解更少的商用车,往往能获得更优解。现有工作通过在策略参数空间优化\cite{tusimple2023aiday}、在不同自车--周车意图组合情景下联合优化\cite{li2023marc}、或引入同伦轨迹构型\cite{degroot2024topology}等方式开展时空联合规划。

      \paragraph{(4) 小结与对云端预测性决策的启示}
      综上,单车自动驾驶的行为决策与运动规划研究通过预测--决策耦合、情景分支、风险预算与博弈推演等方式较好地处理了自车与周车的短期交互。然而,多数方法仍主要面向较短规划时域与较高更新频率,导致对更长时域的不确定性传播、交通流级约束与执行器/车辆动力学边界的考虑不足。云端预测性决策可在更全局的观测与更长时域的推演能力支持下,提供低频长时域的预测性建议,并与车端高频安全控制形成互补闭环。

    \subsubsection{云控预测性决策与规划研究现状}
      \label{subsub:cloud_control}

      上一小节指出,单车自动驾驶决策与规划虽能较好处理短期交互,但受限于\textbf{短时域、局部可观测与算力约束},在更长时域的不确定性传播、交通流级一致性与可验证安全约束方面仍存在不足。云控系统通常将车端原有的单闭环信息流扩展为“\textbf{车端闭环 + 云端闭环}”的双闭环结构:车辆仍在本地执行实时感知—决策—规划—控制闭环;同时云端/边缘侧基于更广域的全局感知与计算资源并行求解长时域决策规划,向车端下发预测性行驶建议、约束或目标,从而在\textbf{不替代车端安全闭环}的前提下提升整体安全性、通行效率与能耗表现。

      \paragraph{(1) 从协同感知到协同行为/协同规划:车路协同为云控提供“全局可观测性”基础}
      云控体系的有效性首先依赖车路协同感知能力的提升,即车端与路端同时进行感知并融合信息以获得更准确、更完备的环境理解;该方向已形成较成熟的研究体系,并构成后续车路协同与云控决策规划的基础\cite{Huang2025V2X}。在此基础上,研究重心逐步从“协同感知”推进到“协同行为/协同规划”:在局部关键场景或冲突区域内利用车端与基础设施侧的信息互补与计算协同,形成可落地的协同决策与轨迹规划框架。典型地,UniV2X\cite{UniV2X2024}以车端规划为核心,将路端感知信息作为增量观测融入车端规划流程;UniMM-V2X\cite{UniMM_V2X2025}进一步在感知与预测环节解耦并行建模,车端融合路端感知结果,并在决策阶段融合车路两端预测以提升复杂交互理解与规划稳定性。面向更强语义层的协同,V2X-VLM\cite{V2X_VLM2025}引入大模型语义理解能力,将协同信息从“几何/动力学层面的融合”扩展到“感知—语义理解—规划”的端到端趋势,使协同信息服务于意图推断、交互解释与可执行规划生成。除单车层面的协同规划外,面向多车交互的协同策略学习也逐渐成为分支:以路侧单元(Road-Side Unit, RSU)为中心的匝道汇入多智能体强化学习决策\cite{Gan2024MultiVehicle}利用RSU全局感知支撑多车策略学习,并通过两阶段/混合强化学习在效率与安全间权衡;V2X辅助运动规划与控制协同设计\cite{Li2024V2X}则强调在车路信息可用条件下开展规划与控制的一体化协同,使车端执行器能更稳定地跟随云端/路端建议并保持闭环稳定。

      \paragraph{(2) 车--路--云分层协同:集中决策与全局引导两种组织形态}
      随着协同从“局部信息增强”走向更系统性的“车路云”一体化,研究开始关注决策—规划—控制在不同计算层级上的\textbf{功能划分}与\textbf{协同机制}。一种代表性思路是按车端、路端与云端进行分层解耦,通过分层强化学习与多智能体强化学习实现跨层级协同决策与规划控制。例如车路云多智能体HRL-MARL框架\cite{Gao2025V2X}以车端智能体、路端智能体与云端智能体分工协作组织策略学习与执行,体现“局部快速闭环 + 云端长时域全局引导”的体系化设计;在智能路口等冲突密集场景下,多智能体强化学习协同自动驾驶\cite{Yu2025MultiAgent}表明基础设施参与协调时,多车可通过协同策略改善安全性与通行效率。

      总体而言,车路云协同决策规划的组织方式可归纳为两条互补主线:
      \textbf{(i) 冲突点/关键区域的集中决策}:由路侧或边缘云在交叉口、匝道汇入等强耦合区域主导车辆通过次序与速度/轨迹建议;
      \textbf{(ii) 更大空间尺度的云端全局引导}:由云端在路网层级进行宏观调度(路径分配、拥堵消散、分区协调等),并将约束或目标下发到局部执行层实现闭环落地。

      在“集中决策”方向,基础设施主导的感知与决策一体化部署是关键趋势之一。AUTODRAITEC\cite{Kherroubi2024AUTODRAITEC}将感知与决策部署在路侧基础设施侧,并结合学习/强化学习为车辆生成速度建议或控制策略,从而在冲突区域形成更集中一致的协调控制;边缘计算支持的无信号路口防撞调度\cite{Lu2024EdgeComputing}强调利用边缘算力与低时延通信对冲突进行实时调度以降低碰撞风险;在混合交通场景下,边缘云控制路口车队的分层决策与协同控制\cite{Yu2025Hierarchical}突出“边云计算 + 车群协同优化”的分层组织,使路口级决策既能处理瞬时冲突,也能兼顾队列组织与整体效率;面向无信号交叉口等典型冲突场景,车路协同多层决策框架\cite{Cui2025VehicleInfrastructure}通过分层协同决策在安全与效率之间实现系统性权衡。对于匝道合流等冲突区域,云协同合流控制与通信延迟补偿\cite{Yang2025CloudBased}将通信时延显式纳入规划与控制设计,通过延迟补偿提升云控合流策略的可执行性与鲁棒性;进一步扩展到更复杂合流结构,多车道主线与双车道匝道CAV协同控制\cite{Wang2025Cooperative}给出面向多区域/多车协同控制的系统性策略设计,以提升效率与安全。

      在“全局引导”方向,研究开始将路网级优化与局部冲突协调进行统一:城市网络CAV协同重路由与路口协调\cite{Typaldos2025Cooperative}将上层路网rerouting与下层无信号交叉口轨迹协调耦合,体现“云端路网级优化 + 局部最优控制”的体系化闭环;该类方法通常基于云端全局状态与预测,向局部执行层下发路径/时窗/优先级等约束,使局部协调在更大尺度上具备一致性与可解释性。

      \paragraph{(3) 通信与工程约束:决策周期、紧/松耦合与可降级协同}
      车路云协同的系统实现必须面对通信链路的物理约束:车云通信通常通过4G LTE/5G NR实现\cite{xionglu2024_CVIS_survay},相对车端本地计算存在更大时延,对实时控制任务具有天然约束;此外丢包、通信拓扑切换等不可靠因素\cite{xuqing2021unreliable}也会影响云端建议的可用性与稳定性。因此,云端决策规划周期一方面需要足够小以保持“预测性”,另一方面又需显式考虑车云时延、带宽与失效风险,并配套通信失效的降级策略\cite{wang2021platoonmpc}。为缓解通信与计算矛盾,分布式计算与分层控制成为常用工程路线:车云协同分布式计算\cite{dai2023cloudadmm}通过分布式优化框架在不同节点间分摊计算负担;更强调精确性的紧耦合路线往往采用“云端复杂模型/算法 + 车端简单模型/算法”的组合以逼近实时控制\cite{li2023cloudnmpc};更强调鲁棒性的松耦合路线则倾向于让云端执行宏观语义级调度\cite{hailin2022semanticcentered},将高层目标或策略偏好下发至车端,由车端完成安全约束下的实时闭环执行。

      \paragraph{(4) 应用目标:预测性节能与预测性高效通行}
      在应用层面,云控预测性决策规划通常以安全为硬约束,主要聚焦两类典型收益:预测性节能行驶与预测性高效通行。预测性节能行驶利用云端可获得的长时域先验信息(如城市道路红绿灯相位\cite{chenqien2023cpccbus}、高速道路坡度信息\cite{li2022pcctrucks}、周车状态与意图等动静态交通信息)进行速度/换道/队列策略的前瞻优化,并由车端控制器执行。Chen等\cite{chenqien2023cpccbus}显式考虑交叉口排队消散时间影响,通过动态规划求解最优速度序列并下发车端执行,使公交车在不损失通行效率前提下实现多路口连续通过并节能;Wan等\cite{lishuyan2022pccheavytrucks}在高速公路场景下针对重型卡车联合优化速度与挡位,利用坡度信息给出未来\SI{2}{km}的经济车速与档位,并提出路点分割方法(Road Points Segmentation Method, RPSM)降低云端动态规划计算复杂性;Wang等\cite{wangluyao2024cloudpacc}进一步建模前车运动状态,在坡度与前车状态约束下调和“安全跟车速度”与“经济车速”,并通过车端MPC进行速度跟踪以实现预测性车速规划。除纵向节能外,云控也被用于更长时域的换道决策协同:江宇\cite{jiangyu2023lc}在云控架构下仅考虑决策时域内自车最多一次换道决策,并由车端完成换道轨迹规划与执行。云控节能同样可扩展到队列控制:王宙等\cite{wangzhou2023hierarchical}提出云支持的分层式队列预测性巡航控制,云端采用滚动时域动态规划(Rolling Horizon Dynamic Programming, RHDP)优化速度规划,车端部署分布式模型预测控制(Distributed Model Predictive Control, DMPC)以兼顾队列稳定性与节能目标;梅润等\cite{meirun2023platoon}在云端先预测周车长时域速度变化(不考虑周车换道),再在自车最多一次换道假设下联合优化纵向油耗与横向换道安全高效代价,体现“云端粗粒度长时域优化 + 车端可执行轨迹规划”的典型组合。

      预测性高效通行更直接面向通行效率与路网运行质量,并随智能网联车渗透率呈现不同技术形态:在低渗透率下,云端需在多数非受控车辆存在的情况下进行长时域预测并辅助单车决策;在中高渗透率下,多智能网联车辆的协同决策与协同规划成为主导。韩硕\cite{hans2024}针对十字路口场景在云端隐式考虑周车交互,对周车一次换道下的意图概率进行预测并组合成不同概率场景,在概率最高的若干场景下并行进行自车闭环策略树搜索,再利用云端预测性决策结果与粗规划轨迹加速车端规划器求解,体现“云端并行推演 + 车端实时落地”的思路。面向高速匝道场景下大规模异质车辆组的实时协同控制,Shi等\cite{shijia2024rampcloud,shi2023finalstate,shi2024conflictgraph,shi2023ivscheduling}在异质车辆组集总动力学模型基础上,采用动态冲突图树搜索并配合启发式剪枝对主路与匝道车辆通行次序进行实时协同决策,并结合车云时延特性设计鲁棒无模型车辆控制方法,从而提升匝道云控系统在真实通信与异质性条件下的鲁棒性。

      \paragraph{(5) 小结:现实约束下的关键难点}
      尽管上述研究在节能与效率优化上展示了云控的潜在收益,但在较长时间尺度内,实际交通仍将处于智能网联车低渗透率与通信受限并存的状态,这对预测性决策规划的可落地性提出更严苛要求:一方面,面向非受控周车的长时域行为预测往往仍较为简化,车辆参数与意图不确定性难以被充分建模;另一方面,即便引入不确定性,决策时域也常受限于可计算性与通信约束而偏短(例如仅\SI{8}{s}\cite{hans2024}),从而使云端相对单车自动驾驶的增益被弱化并难以形成稳定优势。在实车验证层面,受限于周车动态预测精度,实验往往需要在社会车辆较少时进行或弱化对周车动态的考虑,导致算法在复杂开放交通中的可迁移性与产业化落地面临挑战。综合来看,现阶段云控预测性决策规划的关键研究问题正在从“能否在理想假设下获得收益”转向“在通信不可靠、预测不确定、低渗透率的现实条件下如何构建可降级、可验证、可持续演进的协同决策规划体系”。

  \subsection{车云协同体系架构构建方法研究现状}

    现有研究在车--液耦合机理建模、车端侧倾稳定性控制以及云端预测性决策等方面已分别形成较成熟的技术体系,但缺少将需求—能力—功能—资源进行一致映射并支撑安全闭环论证的统一架构框架。随着汽车智能化与网联化融合发展成为全球共识,液罐车运行不再是孤立的单车问题,而将通过网络与路侧基础设施、边缘/云端计算资源及第三方平台紧密连接,形成智能汽车信息物理系统(Intelligent Vehicle Cyber-Physical System, IVCPS)。在此背景下,液罐车防侧翻任务也将从单车局部技术研究逐步走向车路云一体化体系工程解决方案;本文关注的车云协同液罐车防侧翻(Cloud-based Collaborative Tank Truck Rollover Prevention, CT2ROP)可视为IVCPS中面向安全的典型场景任务之一。

    \paragraph{(1) 车路协同到车联网云服务再到车路云一体化:体系形态的演进主线}
    从工程实践看,路侧基础设施系统性纳入智能车辆运行闭环可追溯至美国加州高速公路先进技术PATH项目(California Partners for Advanced Transportation Technology, PATH)\cite{shladover1991path},其在前向预警、路口决策支持、车路协同编队等方面推动了车路协同的早期落地。路侧设备的引入提升了局域交通信息共享能力,但也往往伴随场景适用范围受限的问题。随后,云服务在2016--2021年间逐步与车联网体系融合,早期主要目标是为联网车辆提供全局管理与信息服务,形成车联网云服务架构以补足单车基于规则的感知与决策能力\cite{chu2021cloudcontrolreview}。这一阶段典型方案包括基于自组织网络的分布式架构\cite{wang2020fogbased}与依托公共服务平台的集中式架构\cite{wang2020cloudconnected}。

    近年来,随着V2X技术发展与单车智能(尤其端到端自动驾驶等)快速进展,网联方案与单车方案呈现加速融合趋势。我国在此基础上提出车路云一体化方案\cite{likeqiang2020cloudcontrol,likeqiang2020principles,icv2023whiteroadcloud},将车辆、路侧基础设施与云端计算/数据资源及第三方平台进行系统级连接,通过云控系统的广域感知、并行计算与全局优化能力为智能网联车辆赋能\cite{ran2023vrc,guo2023greeniov},并进一步考虑车云通信故障下的安全机制\cite{sasaki2016mec}与更丰富的应用场景\cite{jia2016platoonvcps},以实现安全、便捷、高效、绿色、经济的交通体系愿景。IVCPS可视为车路云一体化方案下未来交通系统的典型形态表述。

    \paragraph{(2) 国内外参考架构:从“多视角描述”走向“方法论与工程化落地”}
    针对上述复杂智能交通体系的设计与落地,国内外均形成了具有代表性的参考架构体系。美国交通部牵头在2015年推出网联汽车参考实现架构CVRIA(Connected Vehicle Reference Implementation Architecture)\cite{hejazi2022iottov2x},并在2020年进一步发布协同式智能交通系统参考架构ARC-IT(Architecture Reference for Cooperative and Intelligent Transportation)9.0,以企业、功能、物理、通信、服务、防护六个视角描述系统架构;其在2024年更新至ARC-IT 9.3并重点增强对多式联运旅行、电子交通管理等新概念的支持\cite{usdot2024arcit93}。欧盟地平线2020计划下的C-MobILE项目(Accelerating C-ITS Mobility Innovation and depLoyment in Europe)\cite{cmobile2024project}面向去中心化的协同式智能交通系统(Cooperative Intelligent Transportation System, C-ITS),通过上下文、功能、通信、应用、物理、信息六个视角进行架构描述\cite{lu2018citsdeployment},并提出基于系统建模语言(System Modeling Language, SysML)的架构设计方法\cite{ferrandez2018cmobilecase}。

    在国内,国家智能网联汽车创新中心(China Intelligent and Connected Vehicles Research Institute, CICV)在借鉴国外参考架构的基础上结合我国交通特征与云控系统特点,发布智能汽车信息物理系统参考架构1.0\cite{caicv2019cpsarch10},并在2021年提出包含战略、利益攸关者、服务、系统、安全、防护、标准化七个视角的ICV-7S框架\cite{xu2023mbseivcps},同时参考达索公司提出的需求--功能--逻辑--物理(Requirement--Functional--Logical--Physical, RFLP)流程\cite{dassault2012strategicse}形成方法论支撑,并发布参考架构2.0\cite{caicv2021cpsarch20}。此外,中国汽车工程学会在2024年发布车路云一体化系统云控基础平台功能场景参考架构\cite{sae2024cloudplatformarch10},为IVCPS相关系统的架构设计与平台化落地提供进一步参考。考虑到单车智能能力随AI技术快速演进,智能基础设施建设也将与智慧城市、智慧交通战略协同推进\cite{chu2021cloudcontrolreview},从而对架构的可扩展性与演进性提出更高要求。

    \paragraph{(3) 架构差异与不足:IVCPS的“中心—边缘共治”与多场景涌现性}
    需要指出的是,欧美面向C-ITS/ITS的参考架构多强调去中心化协同与分布式网联自动驾驶,而车路云一体化方案在云端融合了更多决策、控制与路网级协同优化能力,形成\textbf{中心与边缘共治}的IVCPS形态。在国内,IVCPS架构设计目前仍以基于文本的描述为主,虽然开始探索基于模型的架构设计,但多聚焦于单个场景任务的系统工程实践。例如,CICV与清华大学针对云控预测性巡航控制(Cloud-based Predictive Cruise Control, CPCC)系统开展了基于MBSE的架构设计\cite{xu2023mbseivcps},体现了“面向任务场景”的工程化路线。

    然而,从体系视角看,IVCPS包含大量类似CPCC的并行发生且相互交互的典型任务场景,具有显著的开放性、演进性与涌现性,可被视为信息物理体系(Cyber Physical System of Systems, CPSoS)。在这种情形下,传统面向单一系统的“系统思维”架构方法难以完整描述体系全貌,也难以系统性处理跨场景交互与演进过程中的安全闭环论证问题。因此,IVCPS/CT2ROP类场景任务的架构设计需要能够显式刻画多场景迭代、跨场景耦合与持续演进,并将建模、仿真与验证纳入架构闭环。

    \paragraph{(4) 体系工程方法演进:从SoSE到So-MBSoSE的场景迭代建模}
    国际上体系工程(System of Systems Engineering, SoSE)相关方法在国内率先应用于军工领域\cite{wanghailong2023soseoverall,gao2023swarmontology},并逐步推广至更广泛的复杂系统领域\cite{gaoxinghai2017mbdtombe,gao2019digitalsystemmodel}。围绕体系架构建模与体系工程流程建模的研究逐渐兴起\cite{wangweiping2019mbse},并提出体系的波浪模型\cite{fengyimin2024soseframework}与螺旋上升模型\cite{liminghua2021aerospace}以刻画体系演进过程。在此基础上,基于建模与仿真的体系工程(Model and Simulation Based SoS Engineering)被提出\cite{zhanglin2022sosembs},强调将仿真验证从系统设计阶段\cite{liuxinghua2011sysmlsimulink,liutonglei2015tripleredundant,zhangshaojie2018mbsecivilaircraft,weicaise2021airsensor,guoyuliang2023door,qinchangmao2021remoteexpert}纳入体系设计闭环;同时,敏捷体系工程、物联网、智能交通系统等新方法、新工具与新应用也被持续引入体系研究\cite{liuyang2024soseoutlook}。

    面向IVCPS这类开放演进的CPSoS,仅依赖一次性、自上而下的系统级架构描述难以覆盖多场景并行涌现与跨场景交互的复杂性。戚笑景等\cite{qixiaojing2026mbsose}面向可实施、可验证、易扩展的IVCPS体系构建需求,提出分场景的基于模型与仿真的体系工程(Scenario-oriented Model and Simulation Based SoS Engineering, So-MBSoSE)方法,通过\textbf{场景任务迭代}不断完善整体架构并识别跨场景交互,从而支撑复杂大系统的渐进式构建与闭环验证。


\section{本文研究内容}
\label{sec:thesis_content}
  \subsection{本文研究对象}

    本研究以低渗透率下高等级公路云控系统中使用云控预测性规划服务的自动驾驶液罐车与人工驾驶液罐车及为其提供服务的云平台构成的子系统作为研究对象,研究对象如图~\ref{fig:use_case}所示,包含高速公路全路段下场景以及部分其他高等级公路路口场景(本研究暂不考虑红绿灯相位)。

    其中研究场景选择基于以下考虑:液罐车侧翻事故多发于高等级公路尤其是高速公路\cite{wangyang2023},其中较为危险的场景有:上下匝道与连续坡道等需要提前减速并精准控制转向的场景、一般路段及路口等特殊路段与周车交互的场景、道路结构改变与静态障碍等需要避障的场景。故本研究选取的研究场景考虑高速公路全路段,如图~\ref{fig:use_case}所示,包括一般路段与特殊路段(上/下匝道、通过路口/匝道口、大曲率弯道、经过施工、车道数改变等)。且由于液罐车侧翻事故多涉及周车影响,故本研究场景中考虑目标液罐车(简称“自车”)与周车的交互。

    \begin{figure}
      \centering
      \includegraphics[width=0.9\linewidth]{fig/chap1/use_case.jpg}
      % \caption*{图片注释解释说明}
      \caption{研究对象示意图}
      \label{fig:use_case}
    \end{figure} 

    本研究主要针对自动驾驶半挂式液罐车,兼顾可实现主动转向的人工驾驶半挂式液罐车,基于如下考虑:首先,液罐车分为整体式与半挂式,其中半挂式占绝大多数,且载重量大、质心高、控制难度高,相较于整体式更为危险,所以对半挂式液罐车的防侧翻研究将更有实际意义;其次,由于液罐车运输的货物运量大、危险性高、驾驶强度大、对驾驶员技术要求高,本研究认为其未来将被技术成熟、遵守法规、节省驾驶员成本的自动驾驶液罐车取代。故选取自动驾驶半挂式液罐车作为主要研究对象,假设其具有基本的环境感知(对一定范围内的道路结构、周车的感知能力)、决策与规划、闭环控制能力;兼顾转向助力系统支持实现主动转向驾驶辅助功能的人工驾驶半挂液罐车,作为快速推广算法、保障当下危化品液体运输安全的手段。

    本文的研究目标为充分发挥高等级公路云控系统的广域感知和高并行算力优势,将车云协同预测性防侧翻决策与规划与更精确的车端防侧翻控制相结合,在不降低通行效率与经济性的情况下进一步提高液罐车行驶过程的侧翻稳定性并减少陷入危险情况的概率。

  \subsection{研究现状总结及存在的问题}

    通过分析国内外研究现状可知,车云协同下的液罐车防侧翻行驶规划与控制是当前国内外交通与车辆学科的前沿领域,而相关研究方法还相对初步,主要存在以下几方面不足:

    (1)针对车液耦合动力学建模描述问题,现有研究一般采用单摆模型描述液体晃动,但建立在二维平面上的描述虽然能反映液体在车辆侧倾和侧向移动自由度的主要动态,但未考虑不可避免的横纵向耦合流动,尤其是纵向流动对横向动力学的影响。此外,现有研究还忽略了对车辆行驶有重要影响的其他模态,如液体晃动对横摆模态的影响,也尚未考虑不同道路横纵向坡度情况下的稳态液体动力学差异。

    (2)针对车辆防侧翻控制问题,一方面,现有研究在未侧翻模态尚未考虑反应更多关键模态液体晃动模型的加入以及将控制手段结合防侧翻规划,且限于安全问题尚未有进行液罐车防侧翻的实车与模型验证。虽有很多关于流体力学相似性的基础性研究,但同样鲜有关于车辆和液体动力何时同时满足的研究。另一方面,针对危险工况下可能出现的半侧翻模态,目前缺乏对于此状态可控性、动力学特性以及控制方法的研究。

    (3)针对车云协同预测性决策与规划问题,现有研究对于非受控的周车的长时域行为预测过于简单,鲜有考虑这些车辆参数与意图的不确定性,或只考虑了短时域的不确定性进行决策,建模方法只能反映出有限的可能性,得出的结论较为理想,未能充分发挥车云协同的感知与算力优势。且目前尚未有针对液罐车的防侧翻运动规划研究,而现有车端运动规划方法在危险情况下不考虑液体晃动,难以直接应用在液罐车上。

    (4)对于车云协同液罐车防侧翻体系建设而言,目前该体系尚未形成,各类研究重心较为分散,并未有从体系工程角度从车辆控制、规划与车云协同体系的角度共同出发,以整体为优化目标解决问题的尝试。

  \subsection{总体研究方案}

    本研究基于云控系统,提出了车云协同液罐车防侧翻行驶规划与控制架构,旨在通过云控系统的全局感知,使得液罐车可以精确掌握周围的道路结构、临时施工、周车状态与意图等动静态交通信息(外因、人因);通过云控系统的强大并行算力,可以考虑目标液罐车的侧翻特性,结合周围车辆状态与意图信息,进行预测性决策与防侧翻轨迹规划(内因、人因);通过车端的短时域规划与双模态轨迹跟踪防侧翻控制结合横纵向耦合液体晃动模型,提供紧急时刻的兜底保障(内因、人因),以提高系统可靠性与可应用性,对危化品公路运输安全保障的强化具有十分重要的意义。为实现上述研究目标,本文将围绕车云协同液罐车防侧翻行驶规划与控制架构、车云协同预测性决策与防侧翻规划、车端态防侧翻控制与缩比液罐车实验验证等方面开展研究,本研究的总体研究方案如图~\ref{fig:thesis_archi}所示



    \subsubsection*{(1)车云协同液罐车防侧翻行驶架构}

    首先进行本场景下的场景任务体系架构构建:从信息物理体系的视角对研究对象进行分析,解析体系组成,并结合IVCPS体系架构框架的指导采用So-MBSoSE方法构建本场景下的体系架构,形成体系级防侧翻解决方案。最后,采用架构在环逻辑仿真对所构建的架构进行验证,以在验证完成的架构基础上实现车云协同决策与规划算法及车端控制算法。

    \begin{figure}
      \centering
      \includegraphics[width=1.0\linewidth]{fig/chap1/architecture.jpg}
      % \caption*{图片注释解释说明}
      \caption{本文研究框架}
      \label{fig:thesis_archi}
    \end{figure} 

    \subsubsection*{(2)车云协同预测性决策与防侧翻规划}

    在上述车云协同架构指导下,为从源头防止液罐车陷入危险工况,需要结合云端的广域感知和强大并行算力与车端对应急事件的实时响应能力进行车云协同预测性决策与防侧翻轨迹规划。首先,在云端进行低频率考虑周车不确定性交互的预测性行为决策:建立包含周车参数和意图不确定性的图扩散预测世界模型,并根据云端的周车历史信息与自车意图在世界模型进行推演实现智能体训练,考虑最小化20s长时域累计风险且保证行驶效率与舒适性的预测性决策。其次,根据行为决策结果进行考虑车辆简化侧翻特性的运动规划。以LTR为侧翻状态指标,找到该行为决策下使得$‖LTR‖_∞$最小化的换道轨迹,将该轨迹下发车端。最后,在车端建立简化侧翻模型考虑周车实时意图与最可能及最危险两种情况下的高频率的防侧翻避障应急轨迹规划。车辆根据感知结果自主判断云端建议是否可执行,若不可执行,则启动车端应急轨迹规划生成一条考虑侧翻特性的紧急避障轨迹用于跟踪。

    \subsubsection*{(3)车端防侧翻轨迹跟踪控制与模型试验}

    在上述车云协同架构指导下,为保证基本稳定能力,研究三聚焦于车液耦合动力学建模与防侧翻控制。首先进行横纵向耦合液体晃动建模:建立标准工况下的3D液体CFD数据集,并分析其黑盒输入输出特性,在此基础上建立非线性精确动力学模型用于云端构建的快速算法仿真验证平台,以及车端进行液体状态的推演。最后,分析车端控制器对于简化模型的需求,线性化对控制效果影响最大的模态,并将其与半挂车动力学模型结合,建立线性化半挂液罐车模型及其随运行工况的重线性化机制。其次,针对未侧翻模态进行轨迹跟踪抑晃防侧翻控制多目标最优控制问题,并在车端求解。此外,还考虑通过控制器嵌套的方式将算法用于辅助人工驾驶。最后,通过通过构建车液耦合仿真平台和缩比液罐车实验平台作为上述方法验证的手段。编写代码以MATLAB/Simulink为桥梁实现流体力学仿真软件StarCCM+和车辆动力学仿真软件TruckSim的联合仿真;其次构建云控缩比液罐车实验平台,包括液罐车改装、通信链路构建等,并进行实验相似性原则分析,以确保车辆侧翻动力学和流体力学同时满足相似性。最后,依据相似性原则进行试验工况设计并开展试验验证工作。

  \subsection{本文创新点与主要贡献}

    围绕“车云协同液罐车防侧翻行驶规划与控制”这一核心目标,本文主要创新点与贡献概括如下:
    \begin{itemize}
      \item \textbf{体系架构层}:面向云控系统下液罐车防侧翻场景,基于IVCPS框架并采用So-MBSoSE方法构建可实现、可验证的场景任务体系架构,并通过架构在环进行一致性验证,为算法落地提供系统边界与接口约束。
      \item \textbf{云端预测性决策层}:面向低渗透率交通流中周车参数与意图不确定性,构建图扩散交互式预测世界模型并用于长时域(约20\,s)风险累积最小化的预测性决策学习,充分发挥云端广域感知与并行算力优势。
      \item \textbf{防侧翻规划层}:以LTR等侧翻指标刻画车辆侧倾风险,形成与云端行为决策一致的防侧翻换道轨迹生成与下发机制,并在车端增加可执行性判别与应急兜底规划以提升闭环可靠性。
      \item \textbf{车液耦合建模与控制层}:面向液体晃动的横纵向耦合特征,建立能够服务于控制设计与在线重线性化的车液耦合模型,并提出兼顾轨迹跟踪与抑晃防侧翻的多目标最优控制框架。
      \item \textbf{验证层}:构建车液耦合联合仿真与缩比实验平台,结合相似性原则进行工况设计,对车云协同规划与车端控制方法的有效性与可实现性进行验证。
    \end{itemize}

  \subsection{论文结构安排}
    
    本文共分为五章,各章内容安排如下:
    \begin{itemize}
      \item 第1章:引言。阐述研究背景与意义,综述国内外研究现状,明确研究对象、关键问题与总体技术路线。
      \item 第2章:车云协同防侧翻场景任务体系架构设计。基于IVCPS与So-MBSoSE方法构建并验证车云协同液罐车防侧翻行驶架构。
      \item 第3章:车云协同预测性决策与防侧翻规划。构建图扩散交互预测模型与强化学习长时域决策方法,并给出车端应急轨迹规划与仿真验证。
      \item 第4章:车端防侧翻轨迹跟踪控制与模型试验。开展横纵向耦合液体晃动建模、抑晃防侧翻控制设计,并完成联合仿真与缩比试验验证。
      \item 第5章:总结与展望。总结全文研究工作与主要贡献,讨论不足并展望未来研究方向。
    \end{itemize}

