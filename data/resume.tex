% !TeX root = ../thuthesis-example.tex

\begin{resume}

  \section*{个人简历}

  2000 年 12 月 1 日出生于山东省威海市.

  2019 年 9 月考入吉林大学汽车工程学院工业设计专业,2023 年 7 月本科毕业并获得工学学士学位.

  2023 年 9 月免试进入清华大学车辆与运载学院攻读工学硕士至今.


  \section*{在学期间完成的相关学术成果}

  \subsection*{学术论文}

  \begin{achievements}
    \item \textbf{Qi X}, Luo Y G, Gao B L, Zhong W. Anti-rollover Path Tracking Control for an Autonomous Semi-trailer Tank Truck[J]. IEEE Transactions on Intelligent Transportation System, 2025, 26(7):9932-9947.DOI:10.1109/TITS.2025.3563475.
    \item \textbf{戚笑景}, 肖雅月, 钟薇, 高博麟, 王勇, 孙棣华, 李克强. 智能网联汽车信息物理系统分场景MBSoSE建模方法[J]. 机械工程学报, 2026, 62(07):1-25.DOI:10.3901/JME.260143.
    \item \textbf{Qi X}, Zhang Y C, Gao B L. Observing Liquid Sloshing Based on a Multi-Degree-of-Freedom Pendulum Model and Free Surface Fluctuation Sensor[J]. Sensors (Basel). 2023 Oct 30;23(21):8831.
    \item 卢彦博, 肖壮锐, 钟薇, 高博麟, 殷国栋, 王彤, \textbf{戚笑景}. 车路云一体化城市预测性巡航系统架构设计研究[J]. 机械工程学报, 2026.(in press 已被机械工程学报录用).
  \end{achievements}


  \subsection*{专利}

  \begin{achievements}
    \item 高博麟, \textbf{戚笑景}, 刘彦斌, 等. 液罐车罐内液体晃动的建模方法、装置、电子设备及介质: 中国, CN117422009A[P]. 2024-01-19.
    \item 高博麟, \textbf{戚笑景}.自由液面波动传感器及液罐车:中国,CN117451142A [P]. 2024-01-26.
    \item 高博麟, \textbf{戚笑景}.一种计算液罐车车液耦合动力学的联合仿真方法:中国,CN116842860A [P]. 2023-10-03.
    \item 高博麟, \textbf{戚笑景}.液罐车防侧翻控制方法、装置、云端、液罐车及控制系统:中国,CN117601746A [P]. 2023-09-25.
    \item 高博麟, \textbf{戚笑景}.用于复杂信息物理系统设计的 RFL (C\&P) 方法:中国,CN117908844A [P]. 2023-12-20.
    \item 高博麟, \textbf{戚笑景}.液罐车防侧翻控制方法、装置、云端、液罐车及控制系统: PCT 美国,WO2025065992A1 [P]. 2024-01-26.
    \item 高博麟, \textbf{戚笑景}.信息物理孪生概率化周车交互式预测模型识别方法:中国,CN119940685A [P]. 2024-11-22.
    \item 高博麟, \textbf{戚笑景},钟薇,卢彦博.基于车云协同的行驶轨迹规划方法及装置:中国,CN119590445A [P]. 2024-10-29.
    \item 钟薇, \textbf{戚笑景},肖雅月,高博麟,王勇,何雷,等.一种基于体系工程的智能网联汽车信息物理系统架构设计方法:中国,CN120110928A [P]. 2025-01-22.
    \item 高博麟,刘家熙,周光,卢彦博,\textbf{戚笑景},熊宣威.动态数据溯源方法、装置、电子设备及存储介质:中国,CN202410395637.9 [P]. 2024-04-02.
    \item 高博麟, \textbf{戚笑景},钟薇,卢彦博.一种液罐车罐内液体晃动高精度建模方法:中国,[P]. (专利申请号).
    \item 高博麟, \textbf{戚笑景},徐熙标,王怡然,钟薇.多轴转向重载车辆封闭环境轨迹规划与跟踪控制方法、系统:中国,[P]. (专利申请号).
    \item 钟薇, \textbf{戚笑景},何瑞坤,肖壮锐,高博麟.液罐车横纵向耦合自适应防侧翻轨迹跟踪控制方法、系统:中国,[P]. (专利申请号).
  \end{achievements}


  \subsection*{参与项目}

  \begin{achievements}
    \item 科技部国家重点研发计划“智能汽车信息物理系统关键技术研究”, 项目编号:2021YFB2501000, 起止时间: 2023.3-2026.1
    \item 科技部国家重点研发计划“机场室内外全场景自动装卸与智能运输装备研制”,项目编号: 2023YFB4301800, 起止时间: 2025.6-至今
    \item 清华-东风合作项目“商用车整车控制技术开发”,起止时间: 2025.6-至今 
    \item 国家自然科学基金面上项目“半挂式只能液罐车多目标规控一体集成控制方法”, 起止时间: 2025.6-至今
    \item 国重实验室自主课题“车云协同半挂式智能液罐车防侧翻轨迹规划与稳定控制研究”, 起止时间: 2023.3-2026.1
  \end{achievements}


  \subsection*{获得的奖励}

  \begin{achievements}
    \item 清华大学硕士生国家奖学金, 2025.10
    \item 首届HitchOpen世界AI竞速锦标赛全国总冠军, 2025.10
    \item 第三届Onsite自动驾驶挑战赛第六赛道大件运输全国冠军, 2025.5 
    \item 首届车路云一体化挑战赛二等奖, 2024.10
    \item 清华大学车辆学院研究生学术论坛口头报告二等奖, 2024.10
  \end{achievements}


\end{resume}
